% O numero máximo de WED-workers que podem ser executados em paralelo, depende
%da quantidade de conecções que o WED-server consegue manter ativas simultaneamente. Caso contrário, um WED-worker pode
%esperar por uma conecção com o WED-server ou abortar a transação.


%--O que é ?

  WED-SQL é uma implementação do modelo WED-flow que utiliza um sgbd relacional e a linguagem SQL para gerenciar,
definir propriedades e controlar o fluxo de dados de transações longas (Long-lived Transactions). Por utilizar
como base tecnologias consolidadas e amplamente adotadas no universo da computação, a WED-SQL visa difundir a adoçao
do paradigma WED-flow na modelagem de processos de negócio , além de criar uma ferramenta robusta, facilmente
escalável, simples de ser utilizada e que permita flexibilidade na especificação de controle de fluxos.

%--Como funciona ? (wed-server, wed-worker, bg_worker)
%webservice, fault tolerance, keep alive transaction, exception management

  Com o objetivo de  ser utilizada em um ambiente distribuido, mais especificamente por meio de web-services, e também
devido à natureza das transações longas, a WED-SQL foi construida utilizando a arquitetura cliente-servidor. A componente
servidor, ou WED-server, é responsável por fazer o controle de fluxo de dados de acordo com as WED-conditions definidas
para um determinado WED-flow, disparando as WED-triggers conforme necessário. Já a componente cliente, ou WED-worker (
quem sabe até WED-service), é quem efetivamente realiza as WED-transitions.

%postgresl catalog driven, mvcc, extensions
%aproveitar arcabouço transacional, linguagem alto nivel
  O WED-server é, basicamente, uma extensão para o sgbd Postgresql composta de triggers, tabelas de controle e "stored
procedures". A escolha do Postgresql como base do WED-server se deu por diversos motivos, tais quais, ser de código aberto, 
implemetação modular baseada em catálogs (system catalog driven), facilidade de extender suas funcionalidades por meio de
código C e tambem em liguagens de alto nível como Python e Perl. Por ser de código aberto, a licensa do Postgresql garante
que o software possa ser modificado e redistribuido, além de que, o acesso ao código fonte possibilita uma melhor compreensão
dos mecanismos interno, garantido uma implementação mais robusta e eficiente do paradigma WED-flow. O postgresql armazena seus
dados de controle em tabelas que são acessíveis aos usuários, que são chamadas de catálogos de sistema, viabilizando a 
criação de modulos para extender suas funcionalidades. Dados de transações podem ser facilmente obtidos por meio desses 
catálogos de sistema, o que é partircularmente útil no caso do WED-server. Dada a natureza dinâmica do modelo WED-flow,
a utilização de uma linguagem de programação de alto nível, como Python, é fundamental para que a expressividade desse
modelo seja implementada de forma plena. Do ponto de vista de implementação, o WED-server contém código Python, SQL, C e 
aproveita o arcabouço transacional clássico (propriedades ACID, controle de concorrência e etc) fornecido pelo Postgresql.

  Já o WED-worker tem a função de conectar-se ao WED-server para efetuar as WED-transitions. Cada WED-transition é associada
a um ou mais WED-workers, dependendo da demanda de trabalho gerada pelo WED-server, e cada WED-worker é especializado
em realizar uma WED-transition específica. A quantidade máxima de WED-workers trabalhando simultaneamente é limitada
apenas pela quantidade de conecções que um dado WED-server é capaz de manter abertas.
  
  %ESTRUTURA (semantica da aplicacao, workflow)
  Um WED-flow é um conjunto de WED-triggers, que associam WED-conditions, definidas em cima de 
um subconjunto de WED-attributes, e WED-transitions.
  Embora seja possível definir multiplos WED-flows em uma única base de dados, o WED-server permite que cada um deles 
esteja localizado em uma base de dados distinta, de acordo com um determinado significado semântico. Com isso também é 
possível isolar atributos que não devem ser compartilhados entre diferentes WED-flows. 
  
  %ER
  O seguinte diagrama ER representa a estrutura de dados implementada como uma base de dados no WED-server:
  ---picture---

  %tabelas(wed_flow,wed_attr,wed_trig,wed_trace,job_pool)
  Os WED-attributes são definidos na tabela wed_attr por um nome e, opcionalmente, por um valor padrão representados 
pelas colunas "aname" e "adv" respectivamente. Cada WED-attribute é identificado univocamente por seu nome que também é 
a chave primária da tabela. Ao criar-se um novo WED-attribute, ou seja, ao inserir-se uma nova linha na tabela wed_attr, 
uma coluna de mesmo nome será automaticamente criada na tabela wed_flow.
  A tabela wed_trig contém as WED-triggers, ou seja, cada linha representa a associação de uma WED-condition com uma 
WED-transition. Possui os seguintes atributos:

  - tgid: identificador único de uma WED-trigger;
  - tgname: atributo opcional que pode ser utilizado para dar um nome à WED-trigger;
  - enabled: permite que a WED-trigger seja desativada;
  - trname: identificador único da WED-transition associada.
  - cname: atributo opcional que pode ser utilizado para dar um nome à WED-condition associada;
  - cpred: predicado da WED-condition associada. Aceita qualquer predicado valido na claúsula WHERE em SQL;
  - cfinal: utilizada para indicar qual é a condição final. Embora apenas uma WED-condition possa ser marcada como final,
           o operado lógico OU pode ser utilizado em seu predicado para definir-se multiplos WED-states finais.
  - timeout: tempo limite para que um WED-worker finalize a WED-transition representada por "trname".

  O histórico de execução das WED-transition fica armazenado na tabela wed_trace, composta dos seguintes atributos:
  
  - wid: referência ao identificador da instancia do WED-flow;
  - state: WED-state, em formato json, que disparou as WED-transitions listadas em "trf";
  - trf: lista de WED-transitions disparadas pelo WED-state em "state";
  - trw: WED-transition que gravou o WED-state representado por "state". Um valor nulo indica o WED-state inicial;
  - status: Indica qual o estado do WED-state em "state". Os possíveis valores são "F","E" ou "R" que indicam que o WED-state
           é final, excessão ou regular, respectivamente.
  - tstmp: indica o momento em que ocorreu o registro. Pode se recuperar a história de execução de uma instância ordenando-se
          as entradas nessa tabela por essa coluna.
  
  A tabela job_pool contém entradas referentes as WED-transitions pendentes que precisam ser executadas pelos WED-workers.
Suas colunas são:
  
  - wid: referência ao identificador da instancia do WED-flow;
  - tgid: referência a WED-trigger que disparou a WED-transition "trname";
  - trname: nome da WED-transition a ser executada;
  - lckid: parâmetro opcional que pode ser utilizado pelos WED-workers para se identificarem. Futuramente, pode ser utilizado
          para fins de autenticação e validação dos WED-workers;
  - timeout: tempo limite para a execução da WED-transition (tempo limite da transação). É uma cópia da coluna de mesmo
            nome da tabela wed_trig;
  - payload: WED-state, em formato json, que disparou a referida WED-transition. Pode ser utilizado, por exemplo, por
            WED-workers que executam WED-transitions associadas a WED-conditions que possuem o operador lógico "OR" em 
            seu predicado.
  
  Finalmente, a tabela wed_flow é o ponto de entrada para inicializar-se uma nova instância. Essa tabela é criada dinâmicamente
de acordo com os WED-attributes definidos na tabela wed_attr. Cada entrada em wed_attr corresponde a uma coluna em wed_flow.
Suas entradas são o WED-state atual de cada instância, ou seja, são tuplas formadas por um identificador, representado
na coluna "wid", e os WED-attributes. 
  

 
  %Definir novo WED-flow (1 por db, atributos de interesse)
  Para se criar um novo WED-flow é preciso definir um conjunto de WED-atributes e um conjunto de WED-triggers associando
WED-transitions à WED-conditions. Essa definição, por ora, é expressa em SQL (futuramente em WSQL)
  
  
%ALGORITMO (trava update, uma instancia por linha, )

%DETALHES DE FUNCIONAMENTO (balanço: coneccoes x demanda,payload pode nao ser o estado atual)

%escalabilidade (horizontal, vertical, consistencia transacional, pg_shard)
 
%TRABALHOS FUTUROS (linguagem, gerenciador de conecções, remocao de attr)





  

 
    
