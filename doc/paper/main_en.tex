
%% bare_conf.tex
%% V1.3
%% 2007/01/11
%% by Michael Shell
%% See:
%% http://www.michaelshell.org/
%% for current contact information.
%%
%% This is a skeleton file demonstrating the use of IEEEtran.cls
%% (requires IEEEtran.cls version 1.7 or later) with an IEEE conference paper.
%%
%% Support sites:
%% http://www.michaelshell.org/tex/ieeetran/
%% http://www.ctan.org/tex-archive/macros/latex/contrib/IEEEtran/
%% and
%% http://www.ieee.org/

%%*************************************************************************
%% Legal Notice:
%% This code is offered as-is without any warranty either expressed or
%% implied; without even the implied warranty of MERCHANTABILITY or
%% FITNESS FOR A PARTICULAR PURPOSE! 
%% User assumes all risk.
%% In no event shall IEEE or any contributor to this code be liable for
%% any damages or losses, including, but not limited to, incidental,
%% consequential, or any other damages, resulting from the use or misuse
%% of any information contained here.
%%
%% All comments are the opinions of their respective authors and are not
%% necessarily endorsed by the IEEE.
%%
%% This work is distributed under the LaTeX Project Public License (LPPL)
%% ( http://www.latex-project.org/ ) version 1.3, and may be freely used,
%% distributed and modified. A copy of the LPPL, version 1.3, is included
%% in the base LaTeX documentation of all distributions of LaTeX released
%% 2003/12/01 or later.
%% Retain all contribution notices and credits.
%% ** Modified files should be clearly indicated as such, including  **
%% ** renaming them and changing author support contact information. **
%%
%% File list of work: IEEEtran.cls, IEEEt#ran_HOWTO.pdf, bare_adv.tex,
%%                    bare_conf.tex, bare_jrnl.tex, bare_jrnl_compsoc.tex
%%*************************************************************************

% *** Authors should verify (and, if needed, correct) their LaTeX system  ***
% *** with the testflow diagnostic prior to trusting their LaTeX platform ***
% *** with production work. IEEE's font choices can trigger bugs that do  ***
% *** not appear when using other class files.                            ***
% The testflow support page is at:
% http://www.michaelshell.org/tex/testflow/



% Note that the a4paper option is mainly intended so that authors in
% countries using A4 can easily print to A4 and see how their papers will
% look in print - the typesetting of the document will not typically be
% affected with changes in paper size (but the bottom and side margins will).
% Use the testflow package mentioned above to verify correct handling of
% both paper sizes by the user's LaTeX system.
%
% Also note that the "draftcls" or "draftclsnofoot", not "draft", option
% should be used if it is desired that the figures are to be displayed in
% draft mode.
%
\documentclass[conference]{IEEEtran}
\usepackage{blindtext, graphicx}
% Add the compsoc option for Computer Society conferences.
%
% If IEEEtran.cls has not been installed into the LaTeX system files,
% manually specify the path to it like:
% \documentclass[conference]{../sty/IEEEtran}





% Some very useful LaTeX packages include:
% (uncomment the ones you want to load)


% *** MISC UTILITY PACKAGES ***
%
%\usepackage{ifpdf}
% Heiko Oberdiek's ifpdf.sty is very useful if you need conditional
% compilation based on whether the output is pdf or dvi.
% usage:
% \ifpdf
%   % pdf code
% \else
%   % dvi code
% \fi
% The latest version of ifpdf.sty can be obtained from:
% http://www.ctan.org/tex-archive/macros/latex/contrib/oberdiek/
% Also, note that IEEEtran.cls V1.7 and later provides a builtin
% \ifCLASSINFOpdf conditional that works the same way.
% When switching from latex to pdflatex and vice-versa, the compiler may
% have to be run twice to clear warning/error messages.






% *** CITATION PACKAGES ***
%
%\usepackage{cite}
% cite.sty was written by Donald Arseneau
% V1.6 and later of IEEEtran pre-defines the format of the cite.sty package
% \cite{} output to follow that of IEEE. Loading the cite package will
% result in citation numbers being automatically sorted and properly
% "compressed/ranged". e.g., [1], [9], [2], [7], [5], [6] without using
% cite.sty will become [1], [2], [5]--[7], [9] using cite.sty. cite.sty's
% \cite will automatically add leading space, if needed. Use cite.sty's
% noadjust option (cite.sty V3.8 and later) if you want to turn this off.
% cite.sty is already installed on most LaTeX systems. Be sure and use
% version 4.0 (2003-05-27) and later if using hyperref.sty. cite.sty does
% not currently provide for hyperlinked citations.
% The latest version can be obtained at:
% http://www.ctan.org/tex-archive/macros/latex/contrib/cite/
% The documentation is contained in the cite.sty file itself.






% *** GRAPHICS RELATED PACKAGES ***
%
\ifCLASSINFOpdf
  % \usepackage[pdftex]{graphicx}
  % declare the path(s) where your graphic files are
  % \graphicspath{{../pdf/}{../jpeg/}}
  % and their extensions so you won't have to specify these with
  % every instance of \includegraphics
  % \DeclareGraphicsExtensions{.pdf,.jpeg,.png}
\else
  % or other class option (dvipsone, dvipdf, if not using dvips). graphicx
  % will default to the driver specified in the system graphics.cfg if no
  % driver is specified.
  % \usepackage[dvips]{graphicx}
  % declare the path(s) where your graphic files are
  % \graphicspath{{../eps/}}
  % and their extensions so you won't have to specify these with
  % every instance of \includegraphics
  % \DeclareGraphicsExtensions{.eps}
\fi
% graphicx was written by David Carlisle and Sebastian Rahtz. It is
% required if you want graphics, photos, etc. graphicx.sty is already
% installed on most LaTeX systems. The latest version and documentation can
% be obtained at: 
% http://www.ctan.org/tex-archive/macros/latex/required/graphics/
% Another good source of documentation is "Using Imported Graphics in
% LaTeX2e" by Keith Reckdahl which can be found as epslatex.ps or
% epslatex.pdf at: http://www.ctan.org/tex-archive/info/
%
% latex, and pdflatex in dvi mode, support graphics in encapsulated
% postscript (.eps) format. pdflatex in pdf mode supports graphics
% in .pdf, .jpeg, .png and .mps (metapost) formats. Users should ensure
% that all non-photo figures use a vector format (.eps, .pdf, .mps) and
% not a bitmapped formats (.jpeg, .png). IEEE frowns on bitmapped formats
% which can result in "jaggedy"/blurry rendering of lines and letters as
% well as large increases in file sizes.
%
% You can find documentation about the pdfTeX application at:
% http://www.tug.org/applications/pdftex





% *** MATH PACKAGES ***
%
%\usepackage[cmex10]{amsmath}
% A popular package from the American Mathematical Society that provides
% many useful and powerful commands for dealing with mathematics. If using
% it, be sure to load this package with the cmex10 option to ensure that
% only type 1 fonts will utilized at all point sizes. Without this option,
% it is possible that some math symbols, particularly those within
% footnotes, will be rendered in bitmap form which will result in a
% document that can not be IEEE Xplore compliant!
%
% Also, note that the amsmath package sets \interdisplaylinepenalty to 10000
% thus preventing page breaks from occurring within multiline equations. Use:
%\interdisplaylinepenalty=2500
% after loading amsmath to restore such page breaks as IEEEtran.cls normally
% does. amsmath.sty is already installed on most LaTeX systems. The latest
% version and documentation can be obtained at:
% http://www.ctan.org/tex-archive/macros/latex/required/amslatex/math/





% *** SPECIALIZED LIST PACKAGES ***
%
\usepackage{algorithmic}
\usepackage{algorithm}
%\floatname{algorithm}{Algoritmo}
\renewcommand{\algorithmicrequire}{\textbf{Input:}}
\renewcommand{\algorithmicensure}{\textbf{Output:}}
% algorithmic.sty was written by Peter Williams and Rogerio Brito.
% This package provides an algorithmic environment fo describing algorithms.
% You can use the algorithmic environment in-text or within a figure
% environment to provide for a floating algorithm. Do NOT use the algorithm
% floating environment provided by algorithm.sty (by the same authors) or
% algorithm2e.sty (by Christophe Fiorio) as IEEE does not use dedicated
% algorithm float types and packages that provide these will not provide
% correct IEEE style captions. The latest version and documentation of
% algorithmic.sty can be obtained at:
% http://www.ctan.org/tex-archive/macros/latex/contrib/algorithms/
% There is also a support site at:
% http://algorithms.berlios.de/index.html
% Also of interest may be the (relatively newer and more customizable)
% algorithmicx.sty package by Szasz Janos:
% http://www.ctan.org/tex-archive/macros/latex/contrib/algorithmicx/




% *** ALIGNMENT PACKAGES ***
%
%\usepackage{array}
% Frank Mittelbach's and David Carlisle's array.sty patches and improves
% the standard LaTeX2e array and tabular environments to provide better
% appearance and additional user controls. As the default LaTeX2e table
% generation code is lacking to the point of almost being broken with
% respect to the quality of the end results, all users are strongly
% advised to use an enhanced (at the very least that provided by array.sty)
% set of table tools. array.sty is already installed on most systems. The
% latest version and documentation can be obtained at:
% http://www.ctan.org/tex-archive/macros/latex/required/tools/


%\usepackage{mdwmath}
%\usepackage{mdwtab}
% Also highly recommended is Mark Wooding's extremely powerful MDW tools,
% especially mdwmath.sty and mdwtab.sty which are used to format equations
% and tables, respectively. The MDWtools set is already installed on most
% LaTeX systems. The lastest version and documentation is available at:
% http://www.ctan.org/tex-archive/macros/latex/contrib/mdwtools/


% IEEEtran contains the IEEEeqnarray family of commands that can be used to
% generate multiline equations as well as matrices, tables, etc., of high
% quality.


%\usepackage{eqparbox}
% Also of notable interest is Scott Pakin's eqparbox package for creating
% (automatically sized) equal width boxes - aka "natural width parboxes".
% Available at:
% http://www.ctan.org/tex-archive/macros/latex/contrib/eqparbox/





% *** SUBFIGURE PACKAGES ***
%\usepackage[tight,footnotesize]{subfigure}
% subfigure.sty was written by Steven Douglas Cochran. This package makes it
% easy to put subfigures in your figures. e.g., "Figure 1a and 1b". For IEEE
% work, it is a good idea to load it with the tight package option to reduce
% the amount of white space around the subfigures. subfigure.sty is already
% installed on most LaTeX systems. The latest version and documentation can
% be obtained at:
% http://www.ctan.org/tex-archive/obsolete/macros/latex/contrib/subfigure/
% subfigure.sty has been superceeded by subfig.sty.



%\usepackage[caption=false]{caption}
%\usepackage[font=footnotesize]{subfig}
% subfig.sty, also written by Steven Douglas Cochran, is the modern
% replacement for subfigure.sty. However, subfig.sty requires and
% automatically loads Axel Sommerfeldt's caption.sty which will override
% IEEEtran.cls handling of captions and this will result in nonIEEE style
% figure/table captions. To prevent this problem, be sure and preload
% caption.sty with its "caption=false" package option. This is will preserve
% IEEEtran.cls handing of captions. Version 1.3 (2005/06/28) and later 
% (recommended due to many improvements over 1.2) of subfig.sty supports
% the caption=false option directly:
%\usepackage[caption=false,font=footnotesize]{subfig}
%
% The latest version and documentation can be obtained at:
% http://www.ctan.org/tex-archive/macros/latex/contrib/subfig/
% The latest version and documentation of caption.sty can be obtained at:
% http://www.ctan.org/tex-archive/macros/latex/contrib/caption/




% *** FLOAT PACKAGES ***
%
%\usepackage{fixltx2e}
% fixltx2e, the successor to the earlier fix2col.sty, was written by
% Frank Mittelbach and David Carlisle. This package corrects a few problems
% in the LaTeX2e kernel, the most notable of which is that in current
% LaTeX2e releases, the ordering of single and double column floats is not
% guaranteed to be preserved. Thus, an unpatched LaTeX2e can allow a
% single column figure to be placed prior to an earlier double column
% figure. The latest version and documentation can be found at:
% http://www.ctan.org/tex-archive/macros/latex/base/



%\usepackage{stfloats}
% stfloats.sty was written by Sigitas Tolusis. This package gives LaTeX2e
% the ability to do double column floats at the bottom of the page as well
% as the top. (e.g., "\begin{figure*}[!b]" is not normally possible in
% LaTeX2e). It also provides a command:
%\fnbelowfloat
% to enable the placement of footnotes below bottom floats (the standard
% LaTeX2e kernel puts them above bottom floats). This is an invasive package
% which rewrites many portions of the LaTeX2e float routines. It may not work
% with other packages that modify the LaTeX2e float routines. The latest
% version and documentation can be obtained at:
% http://www.ctan.org/tex-archive/macros/latex/contrib/sttools/
% Documentation is contained in the stfloats.sty comments as well as in the
% presfull.pdf file. Do not use the stfloats baselinefloat ability as IEEE
% does not allow \baselineskip to stretch. Authors submitting work to the
% IEEE should note that IEEE rarely uses double column equations and
% that authors should try to avoid such use. Do not be tempted to use the
% cuted.sty or midfloat.sty packages (also by Sigitas Tolusis) as IEEE does
% not format its papers in such ways.





% *** PDF, URL AND HYPERLINK PACKAGES ***
%
%\usepackage{url}
% url.sty was written by Donald Arseneau. It provides better support for
% handling and breaking URLs. url.sty is already installed on most LaTeX
% systems. The latest version can be obtained at:
% http://www.ctan.org/tex-archive/macros/latex/contrib/misc/
% Read the url.sty source comments for usage information. Basically,
% \url{my_url_here}.





% *** Do not adjust lengths that control margins, column widths, etc. ***
% *** Do not use packages that alter fonts (such as pslatex).         ***
% There should be no need to do such things with IEEEtran.cls V1.6 and later.
% (Unless specifically asked to do so by the journal or conference you plan
% to submit to, of course. )

%\usepackage[brazilian]{babel}
\usepackage[utf8]{inputenc}
\usepackage[T1]{fontenc}
\usepackage{fancyvrb}

% correct bad hyphenation here
\hyphenation{op-tical net-works semi-conduc-tor}


\begin{document}
%
% paper title
% can use linebreaks \\ within to get better formatting as desired
\title{WED-SQL (Draft)}


% author names and affiliations
% use a multiple column layout for up to three different
% affiliations
\author{\IEEEauthorblockN{Bruno Padilha, João E. Ferreira}
\IEEEauthorblockA{Departamento de Ciência da Computação\\
IME-USP\\
Rua do Matão 1010, 05508-090\\
São Paulo, SP, Brasil \\
brunopadilha@usp.br, jef@ime.usp.br}
\and
\IEEEauthorblockN{Calton Pu}
\IEEEauthorblockA{CERCS, Georgia Institute of Technology\\
266 Ferst Drive, 30332-0765\\
Atlanta, GA, USA\\
calton@cc.gatech.edu}}

% conference papers do not typically use \thanks and this command
% is locked out in conference mode. If really needed, such as for
% the acknowledgment of grants, issue a \IEEEoverridecommandlockouts
% after \documentclass

% for over three affiliations, or if they all won't fit within the width
% of the page, use this alternative format:
% 
%\author{\IEEEauthorblockN{Michael Shell\IEEEauthorrefmark{1},
%Homer Simpson\IEEEauthorrefmark{2},
%James Kirk\IEEEauthorrefmark{3}, 
%Montgomery Scott\IEEEauthorrefmark{3} and
%Eldon Tyrell\IEEEauthorrefmark{4}}
%\IEEEauthorblockA{\IEEEauthorrefmark{1}School of Electrical and Computer Engineering\\
%Georgia Institute of Technology,
%Atlanta, Georgia 30332--0250\\ Email: see http://www.michaelshell.org/contact.html}
%\IEEEauthorblockA{\IEEEauthorrefmark{2}Twentieth Century Fox, Springfield, USA\\
%Email: homer@thesimpsons.com}
%\IEEEauthorblockA{\IEEEauthorrefmark{3}Starfleet Academy, San Francisco, California 96678-2391\\
%Telephone: (800) 555--1212, Fax: (888) 555--1212}
%\IEEEauthorblockA{\IEEEauthorrefmark{4}Tyrell Inc., 123 Replicant Street, Los Angeles, California 90210--4321}}




% use for special paper notices
%\IEEEspecialpapernotice{(Invited Paper)}




% make the title area
\maketitle


\begin{abstract}
%\boldmath
%\blindtext[1]
Business process models designed and implemented under WED-flow concepts are true dynamic systems, easy to evolve in a
incremental and adaptive approach and drastically reducing the complexity of exceptions handling when compared to other
more traditional frameworks (like BPEL, Process Algebra and Petri Networks). On the implementation side, however, these
systems must either provide an extra software layer or use some external toolkit, or even a framework, to be able to enforce 
a data-state based transaction control, what includes dealing with semantic exceptions. In this paper we present the 
WED-SQL, a distributed framework that provides a reliable e efficient way to write and implement WED-flow applications.
One of its main attributes is the integration with the PostgreSQL Relational Database Management System, which enables
the WED-SQL to take full advantage of all the transactional properties provided and also benefit from the SQL language
to specify the WED-flow definitions (e.g. WED-conditions, WED-transitions, and so on). 

\end{abstract}
% IEEEtran.cls defaults to using nonbold math in the Abstract.
% This preserves the distinction between vectors and scalars. However,
% if the journal you are submitting to favors bold math in the abstract,
% then you can use LaTeX's standard command \boldmath at the very start
% of the abstract to achieve this. Many IEEE journals frown on math
% in the abstract anyway.

% Note that keywords are not normally used for peerreview papers.
\begin{IEEEkeywords}
WED-flow, Business Process Management, Long Lived Transactions, PostgreSQL, Client-Server Architecture, Distributed Systems
\end{IEEEkeywords}






% For peer review papers, you can put extra information on the cover
% page as needed:
% \ifCLASSOPTIONpeerreview
% \begin{center} \bfseries EDICS Category: 3-BBND \end{center}
% \fi
%
% For peerreview papers, this IEEEtran command inserts a page break and
% creates the second title. It will be ignored for other modes.
\IEEEpeerreviewmaketitle

\section{Introduction}

Most of modern business process management software systems are subject to recurrent structural modifications during its
life cycle. This changes are necessary to incorporate new requisites to the software, once it is often impossible 
to come up with all requirements upfront in the modeling phase of project - some of then only appear after the software
is running in production. Furthermore, it is too expensive and time consuming to try to predict all the possibles situations 
and its outcomes, which ultimately may raise exceptions that must be dealt with. Over time, these modifications leads
to code deterioration, exponentially increasing the maintenance costs and compromising the quality of the system as well 
negatively impacting its overall performance.
\\
\indent  Classic business process specification models, for example WS-BPEL, Process Algebra and Petri Networks, focus primarily
on behavioral interactions and relations between process, pushing data analysis into the background and overlooking its
relevance for the project. Therefore, none of these models exactly fits the needs of design dynamic business process 
management systems, which may result in costly system adaptations and maintenance.
\\
\indent  The WED-flow approach, in contrast with the classical models, capture both the inter-process relations and, with the very
same relevance, the data generated by their interactions. With this in mind, the WED-flow allows systems architects to 
think in terms of data states and transitions, similarly to a finite automata designing. Modifying or adding new business 
rules to the project, which is often translated in creation of new processes, is a simple matter of defining new data states 
and conditions for transitions, thus capturing the true nature of dynamic processes and simplifying an incremental evolution 
of the whole project. Real time exceptions are also easier to be handled and integrated to the project, once this approach
is based on transactional properties, thus providing several recovery mechanisms~\cite{ICWS12}.
\\
%problemas da implementacao e justificativa do WED-SQL
%tolerancia a falhas
\indent In order to provide a standard implementation as well as simplifying its software development, in this work we present 
the WED-SQL, an WED-flow framework. The main purpose of WED-SQL is to implement all WED-flow definitions~\cite{FTPM10} and encapsulate 
the control structure, what includes the transactional control, exceptions handling, data states detection, state transitioning 
and ensure data integrity. This framework effectively is an abstraction layer that allow developers to put more effort
on business logic and come up with better WED-flow projects, also making better use of reusable code. 
\\
\indent The WED-SQL framework has been built inside the PostgreSQL~\cite{PSQL} relational database management system (PostgreSQL RDBMS)
, thereby uses SQL language
to specify WED-flow elements, has full transactional support and it is highly fault tolerant. Once it employs well consolidate 
technologies, widespread in the computer science field, this framework targets to disseminate the adoption of WED-flow on
business process management software design, by means of an easy to use, reliable and scalable tool that is capable to 
provide the flexibility demanded by modern transaction control systems. 
\\
\indent The remaining of this paper is structured as follows: Section II presents WED-SQL general software architecture aspects and
internal data structures. Section III briefly describes how the system works. A discussion about the algorithms involved
is presented in section IV. Following, a detailed description of transactional control and the communication protocol used
for external communication is given in section V. Finally, on section VI, we discuss the next challenges to improve and
expand this framework capabilities.  
  

\section{WED-SQL: general view and architecture}

To better understand what the WED-SQL does, first one needs to understand the basic definitions of the WED-flow paradigm. 
A WED-flow application is composed of a set of WED-attributes $\mathcal{A} = \{a_1,a_2,a_3,\ldots,a_n\}$, a set of WED-states
$\mathcal{S} = \{s_1,s_2,s_3,\ldots,s_m\}$ where $ \forall s_i \in \mathcal{S}, s_i = (v_1,v_2,v_3,\ldots,v_n)$
is a n-tuple where each $i\in[1,n], v_i$ is a value for $a_i \in \mathcal{A}$ and the size of $s_i = |\mathcal{A}|$, a set of
WED-conditions $\mathcal{C}$, a set of WED-transitions $\mathcal{T}$ and a set of WED-triggers $\mathcal{G}$. A WED-condition
$c \in \mathcal{C}$ is a predicate over $\mathcal{S}$, that is, we say that $s \in \mathcal{S}$ satisfies $c$ if the values
of $s$ makes $c$ true. A WED-transition is a function $t: \mathcal{S}\rightarrow\mathcal{S} \in \mathcal{T}$ that receives 
as input a WED-state $s$ and returns another WED-state $s'$. Finally, a WED-trigger $g = (c,t)$ where $g \in \mathcal{G},
c \in \mathcal{C}, t \in \mathcal{T}$ is a 2-tuple that associates a WED-condition $c$ with a WED-transition $t$. Therefore,
an instance of a WED-flow application stars with a initial WED-state $s$ that should match some WED-condition $c$ to fire
the WED-trigger $g$ that, in turn, will initiate a WED-transition $t$, which must complete by some predefined timeout,thus 
generating the next WED-state. This keeps going until a final WED-state is reached, when this instance is then finalized. 
Roughly speaking, a WED-flow application is a transactional data state transitioning machine.
\\
\indent
Aiming to be capable of running in a distributed environment, for instance in a web-services composition way, as well
to broadly support Long Lived Transaction handling, the WED-SQL is based on a Client-Server system architecture and has
two main component thereafter: a server component called WED-server and a client component called WED-worker. The
sever side is responsible for both transactional and flow control, what means that besides coordinating the WED-transitions, 
enforcing individually defined completion time limits, it is also in charge of matching WED-states against WED-conditions 
and WED-trigger firing. On the other hand, the client side is responsible for doing the computation needed to perform each
WED-transition. 
\\
\subsection{WED-worker and WED-server}
Fundamentally, the WED-server is a PostgreSQL extension package consisting of \emph{database triggers}, control tables and 
stored procedures~\cite{NAV}. Choosing PostgreSQL as the WED-server foundation was a decision based on several of its
aspects, such as be an open source project, catalog-driven operation, support for user-written code using low and high level 
programming languages (e.g., C, Python, Perl, TCL and SQL itself).
\\
\indent  WED-SQL directly benefits from the fact that, being an Open-source software, PostgreSQL's license states that it can be used, 
modified, copied and distributed without any kind of payment or agreement. Besides that, having access to the source code
makes development less complex and error prone once it is easier to analyze and, consequently, better comprehend its internal
mechanisms. 
\\
\indent Relational database systems needs to store metadata about internal structure of tables, columns, indexes, etc. This data
is kept on what is know as system catalogs. PostgreSQL not only stores much of its control data in these catalogs
but also base its procedures on then, in what is called a catalog-driven operation. Since they are readily  
available to users as ordinary tables, thus can be modified, it is fairly easy to modify the PostgreSQL behavior. This
property is specially useful to allow the WED-server to manage timed transactions related to WED-transitions and exception 
handling. 
\\
\indent As previously mentioned, users can write PostgreSQL modules using a few different languages. The most simple and straight
forward of then is a SQL dialect called pgSQL, which is basically SQL with variables and flow control structures. Also, modules
can be written in C language and loaded as a dynamic library, providing the user with ultimate logic control and performance.
Despite these two language having native support in PostgreSQL, sometimes pgSQL is not expressive enough when the logic
involved is too complex, while C, albeit fast, is too low level and overloaded with technical implementation details too
perform most of the tasks. The very PostgreSQL documentation recommends using a language other than C unless it is explicit
needed. Fortunately, it also offers support for Python, a much more versatile and expressive programming language than
the two former. Given the dynamic nature of WED-flow models, Python has been proven the perfect tool to do most of the job
on the WED-server side, only recurring to C when critical or where an extra performance is absolutely necessary. And,of
course, SQL statements are every where in both WED-server and WED-workers. 
\\
\indent The role of the WED-workers is to perform the WED-transitions. They must connect to the WED-server and "ask" for a pending
job before opening a new transaction. All WED-transitions are encapsulated in a single transaction with a time limit to 
complete, otherwise they are automatically aborted by the WED-server. Each WED-transition must have at least one WED-worker
associated and, depending on the workload on the server side, can have many more. On the other hand, each WED-worker is 
specialized in execute a specific WED-transition. The maximum number of WED-workers simultaneously transacting depends 
on the number of concurrent connections that the WED-server is able to keep alive.

\subsection{Data Model}

\begin{figure}[!t]
\centering
\includegraphics[width=2.5in]{ER.png}
\caption{Approximate WED-server's Entity-Relationship model}
\label{fig_er}

\end{figure}
 The diagram depicted on figure* represents the data model used by the WED-server. It is not a true Entity-Relationship 
model once the relation between the \emph{WED\_attr} and \emph{WED\_flow} (represented by the arrow in the diagram) is 
managed by a stored procedure rather than by the RDBMS. This is necessary to comply with the dynamic nature of the WED-flow 
approach, which allows the system designer to evolve the system on-the-fly, for instance, including new WED-attributes and 
WED-conditions. In that case, the system's data structure may need to be modified accordingly. 
\\
\indent WED-attributes are represented by the entity \emph{WED\_attr} using the following attributes:
\begin{itemize}
\item $aname$: Name of the WED-attribute and also the primary key;
\item $adv$: Default value for this WED-attribute.
\end{itemize}

The \emph{WED\_flow} entity represents the instances of a WED-flow applications. Besides the identifying attribute (\emph{wid}),
it has a extra one for each WED-attribute defined for a given application, in other words, for each element of \emph{WED\_attr}
there is an attribute $aname$ in \emph{WED\_flow}.
\\
\indent The weak entities \emph{Job\_Pool} and \emph{WED\_trace} results from the relation between the entities \emph{WED\_flow} and 
\emph{WED\_trig}. Identified by the ternary relationship \emph{Match}, \emph{Job\_Pool} represents pending WED-transitions, that is,
the ones awaiting to be processed by a WED-worker, using the following attributes:

\begin{itemize}
\item $wid$: Foreign key that identifies a WED-flow instance;
\item $tgid$: Foreign key that identifies which WED-trigger fired this WED-transition;
\item $trname$: Name of the WED-transition to be performed;
\item $lckid$: Optional parameter used by WED-workers to be able to identify themselves. May be used for authentication
               purposes in the future  
\item $timeout$: References the same name attribute of entity \emph{WED\_trig}. It is the transaction time limit defined for this WED-transition.  
\item $payload$: Represents the WED-state that matched the WED-condition associated with this WED-transition. 
\end{itemize}

Identified by the ternary relationship \emph{Register}, \emph{WED\_trace} represents the history of execution for all WED-flow
instances in a WED-server. Its attributes are :

\begin{itemize}  
\item $wid$: Foreign key that identifies a WED-flow instance;
\item $state$: WED-state that fired the WED-transitions represented by the multivalued attribute \emph{trf};  
\item $trf$: WED-transitions fired by the WED-state in \emph{state} attribute;
\item $trw$: WED-transition that committed the respective WED-state in \emph{state}. A null value indicates a initial WED-state for
             the given WED-flow instance;
\item $status$: Labels \emph{state} accordingly to its status: "F" for a final WED-state, "E" to indicate that an exception has occurred
               and "R" meaning a regular WED-state;
\item $tstmp$: The exact moment when this record was created. Can be used to retrieve the execution history of a WED-flow instance
              in chronological order. 
\end{itemize}

Finally, the \emph{WED\_trig} entity represents the WED-triggers, that is, the associations of WED-conditions with WED-transitions,
using the following attributes:

\begin{itemize}
\item $tgid$: Primary key; 
\item $tgname$: Optional parameter that indicates the name of the WED-trigger; 
\item $enabled$: Allow a WED-trigger to be disabled;
\item $trname$: Unique name of the associated WED-transition;
\item $cname$: Optional, can be used to name the associated WED-condition; 
\item $cpred$: WED-condition's predicated. Accepts the same syntax and tokens allowed in a SQL's \emph{WHERE} clause;
\item $cfinal$: Mark a WED-condition as final. Although only one is allowed, multiple stop conditions can be defined 
               combining then with the logical operator \emph{OR}; 
\item $timeout$: Time limit to perform the WED-transition 
\end{itemize}

This conceptual data model is mapped to five database tables inside the WED-sever, one for each entity previously explained,
using the very same name. The WED\_flow table is, however, special. It will be dynamically modified every time a new record
is added, modified or removed from the WED\_attr table. For example, after inserting a new record into WED\_attr, a new column,
having the same name of this new WED-attribute, will be added to WED\_flow. Each row in WED\_flow represents a distinct application
instance.  

\section{How it works}
\begin{figure}[!t]
\begin{Verbatim}[fontsize=\tiny]
BEGIN;

INSERT INTO wed_attr (aname, adv) VALUES ('a1','ready');
INSERT INTO wed_attr (aname) VALUES ('a2'),('a3');

INSERT INTO wed_trig (tgname,trname,cname,cpred,timeout) 
  VALUES ('t1','tr_a2','c1', $$a1='ready' and (a2 is null)$$, '3d18h');
INSERT INTO wed_trig (tgname,trname,cname,cpred,timeout) 
  VALUES ('t2','tr_a3','c2', $$a1='ready' and (a3 is null)$$, '00:00:30');
INSERT INTO wed_trig (tgname,trname,cname,cpred,timeout) 
  VALUES ('tf','tr_final','cf', $$a1='ready' 
          and (a2 is not null) 
          and (a3 is not null)$$, '00:00:10');
INSERT INTO wed_trig (cpred,cfinal) VALUES ($$a1 <> 'ready'$$, True);

COMMIT;
\end{Verbatim}
\caption{Defining a new WED-flow application}
\label{code_new}
\end{figure}
 
A new WED-flow application is created by defining a set of WED-attributes and a set of WED-triggers that associates a 
WED-condition to a WED-transition. This is accomplished, for the time being, writing these definition in SQL (see an example
on Fig.~\ref{code_new}. In the foreseen future we plan to propose a SQL language extension to express at least the basic WED-flow operations. 
\\
\indent Note that the expression of a WED-condition predicate (\emph{cpred}) uses the same syntax of the SQL clause \emph{WHERE}.
In fact, this expression is employed \emph{as-is} for the WED-server while matching it against a WED-state. When surrounded
by double \$ marks, special symbols like quotes don't need to be escaped in the predicate.
\\
\indent  Also note that, although the application's final condition statement is declared in the same WED\_trig table it does not
have a WED-transition associated, otherwise it would be ignored by the WED-server anyway. To declare a WED-condition as
final, one only need to specify its predicate and set the value of the \emph{cfinal} attribute to true. If the final condition
is missing, all instances of a given application will end up in an exception state. 
\\
\indent We encourage developers to encapsulate all definitions for a new WED-flow application into a single transaction. By doing
so one can prevent a partially defined application in case of syntax errors, which may force the developer to manually truncate
the system tables before running the script again. 
\\
\indent Although it is possible to define multiple applications in a single WED-flow database, the WED-server allows each one of
then to be isolated in a different database. Therefore, applications can be grouped together by some semantic meaning and WED-attributes
that must not be shared between some of then can be isolated. 

\subsection{Creating WED-workers}

Since WED-workers are client applications of the WED-server, they can be written in any programming language that, somehow,
is able to connect to the PostgreSQL database by simply implementing the WED-SQL's communication protocol (described in
details on section~\ref{GT}). The WED-SQL provides a standard WED-worker implementation by means of a Python package named
\emph{BaseWorker}. Figure~\ref{fig_ww} depicts an example of a WED-worker implemented with this package.

\begin{figure}[!t]
\begin{Verbatim}[fontsize=\tiny]
from BaseWorker import BaseClass
import sys

class MyWorker(BaseClass):
    
    #  trname and dbs variables are static in order to conform 
    #with the definition of wed_trans()
        
    trname = 'tr_aaa'
    dbs = 'user=aaa dbname=aaa application_name=ww-tr_aaa'
    wakeup_interval = 5
    
    def __init__(self):
        super().__init__(MyWorker.trname,MyWorker.dbs,MyWorker.wakeup_interval)
    
    # Compute the WED-transition and return a string as the new WED-state, 
    #using the SQL SET clause syntax. Return None to abort transaction
    def wed_trans(self,payload):
        print (payload)
        
        return "a2='done', a3='ready', a4=(a4::integer+1)::text"
        #return None
        
w = MyWorker()

try:
    w.run()
except KeyboardInterrupt:
    print()
    sys.exit(0)
\end{Verbatim}
\caption{Standard WED-worker implementation in Python}
\label{fig_ww}
\end{figure}
The first step to write a WED-worker using the BaseWorker package is to import the module \emph{BaseClass}, which in fact
is an abstract class that implements the communication protocol with the WED-server as well as manages all connections
and transactions involved.  
\\
\indent  Next, a concrete class must implement the abstract method \emph{wed\_trans()} from BaseClass and initialize the following
class attributes: 
\begin{itemize}
\item \emph{trname}: WED-transition name that will be carried out by this WED-worker. Must match the name used on the WED-flow definition;
\item \emph{dbs}: WED-server connection parameters conforming the \emph{psycopg2} driver format. At least the username, database name and
                 the WED-worker name must be provided;
\item \emph{wakeup\_interval}: Time interval in which the WED-worker execution is suspended (sleeping) if there is no new WED-transitions
                              to be performed (explained in details on section~\ref{not}); 
\end{itemize}

The wed\_trans() method works on a specific WED-flow instance at a time. It receives, as a parameter, the WED-state that 
fired this WED-transition and returns a string representing a new WED-state. This return value must comply with the same
syntax used to specify a \emph{SET} clause of a \emph{UPDATE} SQL statement. An empty value can be returned to abort the
transaction.
\\
\indent Finally, the concrete class can be instantiated and executed by invoking its \emph{run()} method.

\section{Algorithms}

After a WED-flow application is defined, loaded into the WED-server and the WED-workers are initialized and ready to run,
new instances are created by inserting a WED-state into the WED\_flow table. The WED-server will assign to a new instance 
a WED-flow unique identifier (wid), which in turn will be used to both track its execution history as well as for transaction
management purposes. If the default values set for the WED-attributes, in the WED\_attr table, happens to match the ones
of a initial WED-state, then a new instance can be created by simply running the following SQL statement on the WED-server:

\begin{Verbatim}[fontsize=\small]

     INSERT INTO wed_flow DEFAULT VALUES;
\end{Verbatim}

When a new instance is initiated or when a WED-worker successfully commits a new WED-state, the WED-server henceforth runs 
its main algorithm (Alg.~\ref{alg1}). It starts by evaluating all WED-conditions predicates against the new WED-state. Then, for
each satisfied predicate, the WED-transition associated with this WED-condition will be fired and registered on the Job\_Pool table
,a notification will also be sent to the respective WED-worker. Besides that, this events will be added to the instance's
tracing record on the WED\_trace table. If a WED-state satisfies the final condition and if there is no pending WED-transitions,
either running or awaiting in the Job\_Pool table, for a given instance, it is then marked as final and cannot be further
modified. In the case when there is no pending WED-transitions and the new WED-state is not final, the instance will be
marked as an exception and a special record will be inserted into Job\_Pool to signal some external recovery mechanism~\cite{ICWS12}.
It is worth mentioning that unless a initial WED-state is also final, it must fire 
at least one WED-transition, otherwise the WED-server will reject the new instance. 
\\
\indent  WED-workers spend most of their idle time suspended, waiting for a notification send by the WED-server when there is work 
to be done. After some time waiting, if not notification has arrived they wake up and ask the WED-server for work (this
behavior will be explained in section~\ref{not}). Either way, before initiate a WED-transition, the WED-worker need to 
"inform" the WED-server of in which instance and what WED-transition it is about to perform. This is done by requesting the
WED-server a special lock. If this lock is granted, then the WED-worker can proceed (See alg.~\ref{alg2})
\\
\indent This lock system is necessary once more than one WED-worker can be simultaneously activated for the same WED-transition.
Therefore, its main role is to avoid the same WED-transition from being performed for the same WED-flow instance at the
same time. Furthermore, it also helps the WED-server to manage transactions. It is wise to assume that no WED-transition
can be committed by a WED-worker not holding a lock.
\\
\indent Once in possession of a lock, the WED-worker must finish its job within the time limit defined for the WED-transition.
The transaction is successfully finalized updating the current WED-state for this instance on the WED\_flow table.

\begin{algorithm}
\caption{WED-server}
\label{alg1}
\begin{algorithmic}[1]
\REQUIRE WED-flow instance $i$
\ENSURE \textbf{true} for a successfully transaction,\\
         \hspace{23pt}\textbf{false} on the otherwise
\STATE $L \leftarrow$ empty WED-transitions list 
\STATE $s \leftarrow$ current WED-state of $i$
\FOR{$tg$ in WED-triggers}
\STATE $c,tr \leftarrow$ $tg($WED-condition$)$,$tg($WED-transition$)$
\IF{$s$ satisfies $c$}
\IF{$c$ is the final condition}
\STATE insert $\_FINAL$ in $L$
\ELSE
\STATE insert $tr$ in $L$
\ENDIF
\ENDIF
\ENDFOR
\IF{$L$ is empty}
\IF{$i$ is a new instance}
\STATE Reject $i$ and abort the transaction
\RETURN \FALSE
\ELSIF{$i$ \textbf{DOES NOT} has any pending WED-transition}
\STATE Set $i$ as an exception
\STATE Insert \_EXCPT into Job\_Pool
\ENDIF
\STATE Update $i$ at WED\_trace
\RETURN \TRUE
\ELSIF{$L$ contains $\_FINAL$}
\IF{$i$ has pending WED-transitions}
\STATE Reject $i$ and abort the transaction
\RETURN \FALSE
\ELSE
\STATE Set $i$ as a final WED-state
\STATE Update $i$ at WED\_trace
\RETURN \TRUE
\ENDIF
\ELSE
\FOR{$tr$ in $L$}
\STATE Insert $tr$ into Job\_Pool
\STATE Notify the WED-worker(s) responsible for $tr$
\RETURN \TRUE
\ENDFOR
\ENDIF

\end{algorithmic}
\end{algorithm}

\begin{algorithm}
\caption{WED-worker}
\label{alg2}
\begin{algorithmic}[1]
\REQUIRE WED-transition's name: \emph{trname}
\ENSURE \textbf{true} for a successfully transaction,\\
         \hspace{23pt}\textbf{false} on the otherwise
\LOOP
\STATE $n \leftarrow$ Null value
\WHILE{$n$ is Null}
\STATE $n \leftarrow wait\_for\_notification(trname,wkup)$
\IF [No notification has arrived after $wkup$ seconds]{$n$ is Null}
\STATE $n \leftarrow $ Ask WED-server for any $trname$ pending WED-transition
\ENDIF
\ENDWHILE
\STATE Initiate the transaction
\IF {got\_lock($n$)} 
\STATE Run the WED-transition $trname$ for a given $i$ WED-flow instance
\STATE Commit the transaction
\RETURN \TRUE
\ELSE
\STATE Abort the transaction
\RETURN \FALSE
\ENDIF  
\ENDLOOP

\end{algorithmic}
\end{algorithm}

\section{Transactional Management}
\label{GT}
Cada instância de um WED-flow pode ser vista como uma Transação Longa (LLT) e, de acordo com ~\cite{SGD87},
suas WED-transitions podem ser vistas como passos Saga. Sendo assim, é função do WED-server garantir a integridade dos estados
de dados, manter a consistência transacional e garantir que todo WED-flow termine ou em um estado final ou em um estado
de exceção e, nesse caso, deve permitir que um passo compensatório possa ser executado. 
\par
Nas tópicos seguintes serão apresentados em detalhes os mecanismos de controle transacional empregados e o protocolo de
comunicação entre múltiplos WED-workers e um WED-server.

%DETALHES DE FUNCIONAMENTO (seriação,locks(),keep connection alive(pessimistic locking),excessoes, tras simultaneas,continuous query,protocolo de comunicaocao, balanço: coneccoes x demanda,payload pode nao ser o estado atual mvcc)
\subsection{Notificações}
\label{not}
%send notify vs continuous query
Ao modelar um sistema cujas operações principais são realizadas com base em eventos de dados, é preciso também conceber-se
algum tipo de mecanismo capaz de capturar tais eventos. Quando esses dados estão armazenados em um SGBD, uma possível 
solução é utilizar um modelo conhecido por \emph{Continuous Query}~\cite{CQ}. Esse método consiste em definir 
uma consulta, com parâmetros específicos para o estado de dados a ser detectado, e executá-la a cada intervalo de tempo.
\par Ao utilizar esse modelo em um sistema sensível ao tempo de resposta na detecção de eventos, é preciso considerar seu
custo computacional que depende tanto da quantidade de eventos distintos quanto da frequência em que ocorrem. Muitos
eventos distintos implica em muitas consultas distintas. Para detectar eventos dos quais não é possível ter uma estimativa razoável
da frequência em que ocorrem, é necessário executar essas consultas no menor intervalo de tempo possível afim de não comprometer  
a capacidade do sistema de responder a eventos em tempo real. Isso tudo acarreta em um considerável desperdício de recursos
computacionais, aumentando o consumo de energia e sobrecarregando o sistema.
\par Dado que o WED-SQL é também um sistema de captura de estados de dados em tempo real, onde o WED-server já realiza
essa tarefa, será que não é possível aproveitar essa computação e notificar cada WED-worker da ocorrência de seu evento
de interesse ? Sim, é possível ! 
\par O PostgreSQL oferece um mecanismo que pode ser utilizado como uma alternativa ao modelo \emph{Continuous Query}, as Notificações Assíncronas.
Por meio do comando \emph{NOTIFY <canal>,<payload>}, o SGDB pode notificar de modo assíncrono um cliente que tenha se registrado, via comando 
\emph{LISTEN <canal>}, em um canal de notificação específico. Além disso, também é possível enviar uma mensagem ao cliente, por meio de "<payload>". Essa mensagem pode
ser, por exemplo, uma cadeia de caracteres no formato \emph{JSON} que represente o estado de dados de uma instância de um WED-flow.
\par A grande vantagem do modelo de Notificações Assíncronas sobre o modelo de Continuous Query é que o cliente recebe a notificação em tempo real
da ocorrência do evento de dados, contanto que o mesmo esteja escutando no canal de notificação nesse momento. Além disso, devido ao fato de o cliente
não precisar verificar a cada certo intervalo de tempo se há alguma nova mensagem, ele pode esperar pela notificação "dormindo", ou seja, sua execução
pode ser suspensa enquanto não há uma nova mensagem, liberando os recursos da máquina para realizar outras tarefas assim como economizando energia.
\par No WED-SQL são utilizados os dois modelos de detecção de eventos de dados. Durante uma execução normal, o WED-server notifica os WED-workers
sempre que uma nova WED-trigger é disparada, ou seja, sempre que surge uma nova tarefa na tabela Job\_Pool. Além disso, a mensagem enviada aos WED-workers
é exatamente a nova tarefa que foi gerada, eliminando a necessidade de se fazer uma busca na tabela Job\_Pool. Cada WED-transition possui um canal exclusivo
de notificações identificado por seu nome definido na tabela WED\_trig. Os WED-workers registram-se em seu respectivo canal e suspendem sua operação até
que sejam acordados com uma notificação. Caso uma notificação seja enviada e não haja ninguém escutando em um determinado canal, a mesma será descartada.
\par Afim de garantir que todas as WED-transitions sejam eventualmente executadas, e de preferência sejam iniciadas no menor intervalo de tempo possível, os
WED-workers não dependem exclusivamente das notificações recebidas para funcionar. De fato, ao inicializar uma nova conexão com o WED-server, o primeiro
passo é procurar por WED-transitions pendentes de execução na tabela Job\_Pool, e só após entrar em estado de suspensão aguardando por notificações. Os 
WED-workers podem ainda ser configurados para "acordar" após um certo intervalo de tempo sem receber novas notificações e verificar por tarefas pendentes,
uma vez que, em se tratando de um sistema potencialmente distribuído, é possível que mensagens possam se perder por falhas na rede de comunicação ou por
outros eventos não detectáveis tanto pelo WED-server quanto pelos WED-workers. 
%Cabe ao projetista garantir que uma trnas nao dependa de atributos que nao foram definidos na cond (attr de interesse). Se a trnas foi disparada é pq a cond for satis. nivel de isolamento permite ...
\par O WED-state recebido pelos WED-workers, seja por meio de notificação ou consultando a tabela Job\_Pool, não representa o estado atual da instância mas
sim o estado de dados no momento em que a WED-transition foi disparada. Sendo assim, dado que a condição de disparo de uma WED-transition é conhecida,
qual a razão de se enviar o estado de dados aos WED-workers ? Eficiência. No caso de WED-conditions definidas por uma conjunção de predicados, ou seja, 
utilizando o operador lógico \emph{OR}, pode ser necessário verificar os valores dos WED-attributes que dispararam a WED-transtion associada. Enviar o estado
de dados aos WED-workers elimina a necessidade de consultas à tabela WED-trace. É muito importante lembrar que, ao definir um WED-worker, não se deve 
considerar atributos que não foram definidos na WED-condition, uma vez que, além de ser um erro de projeto do WED-flow, outras transações executando
em paralelo podem alterá-los a qualquer momento. Cabe ao projetista do WED-flow garantir que uma WED-transition não dependa de WED-atributes que não
foram definidos na WED-condition associada. 


\subsection{Controle transacional e de concorrência}
%wed-trans, why lock ?, types of locks , why advisory(pg_row_locks)
%locks are needed for: 1-keep track of wed-trans; 2-concurrency control between wed-workers (pg_locks, select for update nowait)

Antes que uma WED-transition possa ser executada, é necessário que a respectiva tarefa esteja registrada na tabela Job\_Pool e que
o WED-worker que irá executá-la garanta exclusividade de execução. Isso é efeito requisitando uma trava ao WED-server na tarefa
especificada. Essa trava é necessária por dois motivos: permitir que o WED-server tenha controle sobre o tempo de vida da transação;
gerenciar conflitos entre transações concorrentes.

\par O modelo WED-flow prevê que cada WED-transition tenha um tempo limite de execução. Para atender a esse requisito, o WED-server
precisa identificar a transição que está sendo executada e monitorar o seu tempo de vida. Vale lembrar que a execução de uma
WED-transition está encapsulada em uma única transação. Um modo de forçar as transações a se identificarem é exigir que as mesmas
solicitem uma trava exclusiva ao WED-server. Outra possível solução seria identificar a transação semanticamente de acordo
com o novo WED-state gerado, e nesse caso teríamos dois problemas: o WED-server apenas seria capaz de identificar a transação
no momento da atualização da instância, fazendo com que as transações pudessem ficar ativas muito além de seu tempo limite para
só então serem finalizadas, potencialmente limitando o tempo máximo de execução para todas as transações; seria necessário manter 
um conjunto de possíveis estados de dados pré-definidos, limitando a capacidade do modelo em lidar com exceções. Além do mais,
o WED-SQL utiliza essas travas de execução para resolver problemas de concorrência entre WED-workers e, com isso,
é a solução adotada para a identificação das transações. 


\par Cada WED-transition pode ter multiplos WED-workers idênticos associados. Isso pode ser necessário para atender a demanda 
gerada por um WED-flow com muitas instâncias. Para evitar que mais de um WED-worker execute a mesma WED-transition na mesma
instância, o mesmo deve registrar interesse em executá-la por meio da requisição ao WED-server de uma trava exclusiva na linha
correspondente da tabela Job\_Pool, que é identificada por seu WED-flow id (wid) e a WED-trigger id (tgid) que a disparou.

\par O SGDB PostgreSQL oferece um mecanismo de travas explicitas para aplicações que necessitam exercer um controle mais
refinado sobre os dados acessados. Essas travas podem ser utilizadas tanto a nível de tabelas quanto a nível de tuplas, ou seja,
linhas ou registros das tabelas. No contexto do WED-SQL, apenas as travas a nível de tuplas são utilizadas explicitamente.
\par A trava a nível de tupla mais comumente utilizada é chamada \emph{FOR UPDATE}. Utilizada em conjunto com o comando SQL
\emph{SELECT}, essa trava, de acesso exclusivo, faz com que outras transações concorrentes tenham que esperar ou abortar
ao tentarem acessar a tupla bloqueada. Outro tipo de trava que também pode ser utilizada a nível de tupla são as 
\emph{Advisory Locks}, ou travas semânticas. Para utilizar esse tipo de trava é preciso definir seu significado semântico na aplicação, uma vez que
o PostgreSQL não obriga que sua regra de acesso seja cumprida, ou seja, os registros não são realmente bloqueados. 
\par Por exemplo, no WED-server as WED-transitions pendentes são identificadas pelo par (wid,tgid). Esse identificador é
utilizado pelos WED-workers para requisitar uma \emph{Advisory Lock}, que será concedida apenas à primeira transação que
a solicitar para o dado registro. As demais transações que tentarem acessar o mesmo registro não conseguiram obter a trava,
e estarão livres para tentar executar outras WED-transitions pendentes, embora o registro não esteja de fato bloqueado. Vale
lembrar que, obter essa trava para a instância específica que será atualizada, é mandatório para completar a transação com sucesso.
\par Por que são utilizadas \emph{Advisory Locks} ao invés de \emph{FOR UPDATE} no WED-SQL ? Por dois motivos: primeiramente
pelo modo como o PostgreSQL funciona internamente e, em segundo lugar, pela natureza da operação de  transição de estados
no contexto do WED-SQL. 
\par O principal motivo para não utilizar travas do tipo \emph{FOR UPDATE} é que as informações sobre
quais tuplas estão bloqueadas são armazenadas em disco, e não em memória RAM. Com isso, é necessário utilizar um plugin
externo ao SGBD para recuperar tais informações, necessárias ao WED-server para realizar o controle do tempo de vida
de cada transação. Por outro lado, as informações sobre \emph{Advisory Locks} podem ser obtidas facilmente, por meio de
uma simples consulta a um catálogo interno do PostgreSQL. 
\par Outro motivo para não se utilizar \emph{FOR UPDATE} no controle transacional, é que o registro que seria bloqueado
não é o registro que será atualizado. As travas são obtidas na tabela Job\_Pool, porém as atualizações são executadas na
tabela WED\_flow. Desse modo é mais simples, seguro e eficiente utilizar travas semânticas no WED-server.




\subsection{Isolamento e paralelismo}
O PostgreSQL utiliza o modelo MVCC~\cite{NAV} para manter a integridade dos dados. Isso significa que cada transação enxerga uma "foto"
do banco de dados no momento em que a mesma inicia, garantindo seu isolamento uma vez que não há compartilhamento de dados
parciais entre transações ativas. Nesse modelo de controle de concorrência, travas de leitura não conflitam com travas de escrita.
\par O padrão ISO/ANSI SQL define quatro níveis de isolamento transacional. Em ordem de restrição: \emph{
Read Uncommited, Read Commited, Repeatable Read e Serializable}
\par No nível menos restritivo,\emph{Read Uncommited}, uma transação enxerga modificações realizadas por outras transações
concorrentes que ainda não finalizaram. Esse fenômeno é conhecido como \emph{dirty read}. No PostgreSQL, transações que executam
nesse nível de isolamento são na verdade executadas no nível \emph{Read Commited}, que é o nível mínimo de isolamento possível
na arquitetura MVCC. De qualquer modo, o padrão permite que um SGBD execute suas transações em um nível mais restritivo
de isolamento do que o requisitado.
\par Transações que executam no nível \emph{Read Commited} podem enxergar modificações realizadas por transações concorrentes
que finalizem durante sua execução. Por exemplo, a leitura de uma mesma tupla duas ou mais vezes, dentro de uma mesma transação,
pode retornar valores diferentes. Esse fenômeno é conhecido por \emph{nonrepeatable read}. Esse é o nível padrao do PostgreSQL.
\par Ao contrário do nível anterior, no nível \emph{Repetable Read}, uma transação ativa não enxerga tuplas modificadas
por outras transações que finalizem previamente. Em caso de conflito, apenas uma transação finaliza e as outras são abortadas.
Ainda assim, é possível ocorrer um fenômeno chamado de \emph{phantom read}, no qual um conjunto de tuplas que satisfaça uma
determinada condição difere em duas execuções da mesma consulta na mesma transação. No PostgreSQL esse fenômeno não ocorre.
\par O último e mais restritivo nível de isolamento é o \emph{Serializable}. De acordo com o padrão, um conjunto de transações
concorrentes executando nesse nível deve produzir o mesmo resultado da execução sequencial em alguma ordem. Sendo assim, nenhum
dos fenômenos mencionados anteriormente ocorre. No PostgreSQL, esse nível de isolamento funciona exatamente como o \emph{Repeatable Read}
a menos da introdução de um mecanismo para detectar anomalias que poderiam levar a um resultado não condizente com todas
as possíveis execuções de um conjunto de transações em série.
\par No WED-server, assim como no PostgreSQL, o nível padrão de isolamento transacional é o \emph{Read Commited}. Isso só
é possível graças ao seu mecanismo de detecção de estados de dados que, valendo-se do modelo MVCC, gera uma nova versão das
tabelas relevantes toda vez que uma transação modifica uma instância de um WED-flow. Sendo assim, no caso de duas transações 
concorrentes, por exemplo, não é preciso que uma delas seja abortada, o que ocorreria em níveis mais restritos de isolamento.
Caso uma transação aborte, o WED-server descarta as alterações realizadas. Transações que operam em instância distintas 
são executadas totalmente em paralelo. Além disso, os WED-workers não exergarao as novas WED-transitions 
disparadas até que transação que escreveu o novo estado de dados finalize com sucesso.
\par WED-workers que trabalham simultaneamente na mesma instância executam em paralelo até o momento final, onde necessitam
atualizar o estado de dados dessa instância. Nesse momento, a execução se dará de modo sequencial, uma vez que o PostreSQL
irá conceder uma trava implicita para a tupla que será modificada. Vale notar que a atualização da instância deve ser a 
ultima operação executada pelo WED-worker, limitando o tempo de espera à quantidade de transações pendentes multiplicada 
pelo tempo de execução do comando SQL UPDATE. Caso contrário, a execução ocorre em paralelo (veja a figura~\ref{fig_wf}).
   
\begin{figure}[!t]
\centering
\includegraphics[width=2.5in]{wed-flow2.png}
\caption{Os WED-workers 1 e 2 trabalham na mesma instância e precisam sincronizar a atualização do WED-state na tabela
WED-flow}
\label{fig_wf}
\end{figure}
%escalabilidade (horizontal, vertical, consistencia transacional, pg_shard) 

% needed in second column of first page if using \IEEEpubid
%\IEEEpubidadjcol

% An example of a floating figure using the graphicx package.
% Note that \label must occur AFTER (or within) \caption.
% For figures, \caption should occur after the \includegraphics.
% Note that IEEEtran v1.7 and later has special internal code that
% is designed to preserve the operation of \label within \caption
% even when the captionsoff option is in effect. However, because
% of issues like this, it may be the safest practice to put all your
% \label just after \caption rather than within \caption{}.
%
% Reminder: the "draftcls" or "draftclsnofoot", not "draft", class
% option should be used if it is desired that the figures are to be
% displayed while in draft mode.
%
%\begin{figure}[!t]
%\centering
%\includegraphics[width=2.5in]{myfigure}
% where an .eps filename suffix will be assumed under latex, 
% and a .pdf suffix will be assumed for pdflatex; or what has been declared
% via \DeclareGraphicsExtensions.
%\caption{Simulation Results}
%\label{fig_sim}
%\end{figure}

% Note that IEEE typically puts floats only at the top, even when this
% results in a large percentage of a column being occupied by floats.


% An example of a double column floating figure using two subfigures.
% (The subfig.sty package must be loaded for this to work.)
% The subfigure \label commands are set within each subfloat command, the
% \label for the overall figure must come after \caption.
% \hfil must be used as a separator to get equal spacing.
% The subfigure.sty package works much the same way, except \subfigure is
% used instead of \subfloat.
%
%\begin{figure*}[!t]
%\centerline{\subfloat[Case I]\includegraphics[width=2.5in]{subfigcase1}%
%\label{fig_first_case}}
%\hfil
%\subfloat[Case II]{\includegraphics[width=2.5in]{subfigcase2}%
%\label{fig_second_case}}}
%\caption{Simulation results}
%\label{fig_sim}
%\end{figure*}
%
% Note that often IEEE papers with subfigures do not employ subfigure
% captions (using the optional argument to \subfloat), but instead will
% reference/describe all of them (a), (b), etc., within the main caption.


% An example of a floating table. Note that, for IEEE style tables, the 
% \caption command should come BEFORE the table. Table text will default to
% \footnotesize as IEEE normally uses this smaller font for tables.
% The \label must come after \caption as always.
%
%\begin{table}[!t]
%% increase table row spacing, adjust to taste
%\renewcommand{\arraystretch}{1.3}
% if using array.sty, it might be a good idea to tweak the value of
% \extrarowheight as needed to properly center the text within the cells
%\caption{An Example of a Table}
%\label{table_example}
%\centering
%% Some packages, such as MDW tools, offer better commands for making tables
%% than the plain LaTeX2e tabular which is used here.
%\begin{tabular}{|c||c|}
%\hline
%One & Two\\
%\hline
%Three & Four\\
%\hline
%\end{tabular}
%\end{table}


% Note that IEEE does not put floats in the very first column - or typically
% anywhere on the first page for that matter. Also, in-text middle ("here")
% positioning is not used. Most IEEE journals use top floats exclusively.
% Note that, LaTeX2e, unlike IEEE journals, places footnotes above bottom
% floats. This can be corrected via the \fnbelowfloat command of the
% stfloats package.

\section{Trabalhos Futuros}
%pgBouncer = breaks notify
%wasting connections
%pubsub (rabbitMQ, real event bus)
%pg process size in memory (~9MB)
%file descriptor per process
%(linguagem, gerenciador de conecções, ...)

Na próxima fase desse projeto serão incorporadas duas funcionalidades essenciais para o sucesso do projeto WED-SQL: uma
linguagem declarativa para expressar as operações nativas do WED-flow e um módulo específico para gerenciar as conexões
com o WED-server.
\par Quanto à linguagem, a ideia é criar uma extensão da SQL (de codinome WSQL) que encapsule as operações nativas do WED-flow e integrá-la
ao interpretador do PostgreSQL, o qual será então responsável por traduzir e executar código SQL nativo e outras operações que,
nesse momento, são realizadas por meio de um conjunto de \emph{shell scripts}. Por meio dessa abstração, será possível 
deixar a sintaxe declarativa do WED-server mais próxima do modelo teórico, simplificando sua utilização.
\par Em Linux, um processo pode criar um clone de si mesmo para executar uma porção específica de código de modo independente.
O PostgreSQL, assim como muitos outros serviços que atendem à requisições externas, utiliza essa técnica sempre que 
for estabelecida uma nova conexão com uma aplicação cliente. Ao analisar a utilização de recursos do sistema operacional
, é possível observar que cada novo sub-processo gerado ocupa cerca de 9MB de memória RAM. Como consequência, há um limite
físico no número de conexões que podem ser estabelecidas simultaneamente. No caso do WED-SQL, cada WED-worker precisa 
manter uma conexão ativa com o WED-server para receber as notificações
assíncronas em tempo real, além de estabelecer uma segunda conexão para efetivamente executar a WED-transition. Sendo
assim, para cada WED-worker ativo são criados dois sub-processos do PostgreSQL, o que acaba por consumir em média 18MB
de RAM.
\par As notificações assíncronas do PostgreSQL são baseadas em um modelo de comunicação conhecido por \emph{Publish-Subscribe},
onde quem envia uma mensagem não esta ciente da existência de quem a recebe. Sendo assim, é possível utilizar uma especie
de WED-worker específico que funcionaria com um detector de eventos em tempo real para receber e distribuir as mensagens
para os demais WED-workers, o qual utilizaria uma única conexão com o WED-server, efetivamente dobrando a quantidade
de WED-transitions simultâneas. Há diversas soluções prontas para implementar esse mecanismo, umas delas é o software 
RabbitMQ.


\section{Conclusão}

Desde sua concepção em 2010~\cite{FTPM10}, o paradigma WED-flow não parou de evoluir. Novos estudos, realizados antes
e durante o desenvolvimento deste trabalho, mostram que embora a ferramenta WED-tool~\cite{WT} implemente o WED-flow, 
na prática ela ja não atende aos novos requisitos dessas novas aplicações. 
\par A WED-SQL está sendo estrategicamente elaborada para suprir essa demanda. Sua integração ao SGBD garante que o controle
de fluxo proposto pelo modelo seja executado com propriedades transacionais nativas, além de dispor do mais avançado aparato
para o controle de concorrência assim como garantir a integridade dos dados com as propriedades ACID. A adoção do modelo 
cliente servidor tem por objetivo viabilizar a sua implementação em ambientes distribuídos, uma vez que os WED-workers
tipicamente serão implementados como \emph{Web Services} e múltiplos WED-servers poderão ser utilizados simultaneamente,
seja para manter a disponibilidade, quanto para balanceamento de carga. Ao utilizar uma linguagem declarativa própria 
para a WED-SQL, é possível aproximar sua sintaxe da utilizada no modelo teórico, simplificando sua utilização, e eventualmente
lançar uma versão especial do PostgreSQL otimizada para o WED-flow.
\par Resumindo, trata se de uma ferramenta moderna, altamente escalável, construída em um ambiente transacional cujo
principal objetivo é simplificar a utilização e a implementação de modelos de processos de negócio desenvolvidos sob
o paradigma WED-flow.

%\blindtext

% if have a single appendix:
%\appendix[Proof of the Zonklar Equations]
% or
%\appendix  % for no appendix heading
% do not use \section anymore after \appendix, only \section*
% is possibly needed

% use appendices with more than one appendix
% then use \section to start each appendix
% you must declare a \section before using any
% \subsection or using \label (\appendices by itself
% starts a section numbered zero.)
%


%\appendices
%\section{Apd }
%\blindtext

% use section* for acknowledgement
\section*{Acknowledgment}


The authors would like to thank...


% Can use something like this to put references on a page
% by themselves when using endfloat and the captionsoff option.
\ifCLASSOPTIONcaptionsoff
  \newpage
\fi



% trigger a \newpage just before the given reference
% number - used to balance the columns on the last page
% adjust value as needed - may need to be readjusted if
% the document is modified later
%\IEEEtriggeratref{8}
% The "triggered" command can be changed if desired:
%\IEEEtriggercmd{\enlargethispage{-5in}}

% references section

% can use a bibliography generated by BibTeX as a .bbl file
% BibTeX documentation can be easily obtained at:
% http://www.ctan.org/tex-archive/biblio/bibtex/contrib/doc/
% The IEEEtran BibTeX style support page is at:
% http://www.michaelshell.org/tex/ieeetran/bibtex/
%\bibliographystyle{IEEEtran}
% argument is your BibTeX string definitions and bibliography database(s)
%\bibliography{IEEEabrv,../bib/paper}
%
% <OR> manually copy in the resultant .bbl file
% set second argument of \begin to the number of references
% (used to reserve space for the reference number labels box)
\begin{thebibliography}{1}

\bibitem{FTPM10}
João E. Ferreira, Osvaldo K. Takai, Simon Malkowski e Calton Pu. 
\emph{Reducing exception handling complexity in business process modeling and implementation:
the WED-flow approach.}, Em Proceedings of the 2010 international conference on
On the move to meaningful internet systems - Volume Part I, OTM’10, páginas
150–167. Springer-Verlag, 2010.

\bibitem{ICWS12}
João Eduardo Ferreira, Kelly Rosa Braghetto, Osvaldo Kotaro Takai e
Calton Pu.\emph{ Transactional recovery support for robust exception handling in business process
services.} Em Proceedings of the 19th International Conference on Web Services (ICWS), páginas
303–310, 2012

\bibitem{SGD87}
Hector Garcia Molina, Kenneth Salem.
\emph{Sagas}, Em Proceeding of the 1987 ACM SIGMOD International Conference on Management of Data, páginas 249-259, 1987

\bibitem{CQ}
Ling Liu, Calton Pu, Wei Tang.
\emph{Continual Queries for Internet Scale Event-Driven Information Delivery},
IEEE Trans. on Knowl. and Data Eng. 11, páginas 610–628, 1999

\bibitem{NAV}
Ramez Elmasri, Shamkant B. Navathe.
\emph{Fundamentals of Database Systems}, 6th ed, chapters 1.6.9 and 22.3, 2010

\bibitem{WT}
Marcela Ortega Garcia, Pedro Paulo de S. B. da Silva, Kelly Rosa Braghetto e João Eduardo Ferreira.
\emph{Wed-tool: uma ferramenta para o controle de execução de processos de negócio transacionais},
Proceedings of the 27th Brazillian Symposium on Databases - Demos and Applications Session, página 36, 2012

\bibitem{PSQL}
PostgreSQL, v9.5.x \\
\emph{http://www.postgresql.org/about/}
%\bibitem{IEEEhowto:kopka}
%H.~Kopka and P.~W. Daly, \emph{A Guide to \LaTeX}, 3rd~ed.\hskip 1em plus
%  0.5em minus 0.4em\relax Harlow, England: Addison-Wesley, 1999.

\end{thebibliography}

% biography section
% 
% If you have an EPS/PDF photo (graphicx package needed) extra braces are
% needed around the contents of the optional argument to biography to prevent
% the LaTeX parser from getting confused when it sees the complicated
% \includegraphics command within an optional argument. (You could create
% your own custom macro containing the \includegraphics command to make things
% simpler here.)
%\begin{biography}[{\includegraphics[width=1in,height=1.25in,clip,keepaspectratio]{mshell}}]{Michael Shell}
% or if you just want to reserve a space for a photo:

\begin{IEEEbiography}[{\includegraphics[width=1in,height=1.25in,clip,keepaspectratio]{picture.png}}]{John Doe}
\blindtext
nhaga
\end{IEEEbiography}

% You can push biographies down or up by placing
% a \vfill before or after them. The appropriate
% use of \vfill depends on what kind of text is
% on the last page and whether or not the columns
% are being equalized.

%\vfill

% Can be used to pull up biographies so that the bottom of the last one
% is flush with the other column.
%\enlargethispage{-5in}




% that's all folks
\end{document}

