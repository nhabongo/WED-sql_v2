
%% bare_conf.tex
%% V1.3
%% 2007/01/11
%% by Michael Shell
%% See:
%% http://www.michaelshell.org/
%% for current contact information.
%%
%% This is a skeleton file demonstrating the use of IEEEtran.cls
%% (requires IEEEtran.cls version 1.7 or later) with an IEEE conference paper.
%%
%% Support sites:
%% http://www.michaelshell.org/tex/ieeetran/
%% http://www.ctan.org/tex-archive/macros/latex/contrib/IEEEtran/
%% and
%% http://www.ieee.org/

%%*************************************************************************
%% Legal Notice:
%% This code is offered as-is without any warranty either expressed or
%% implied; without even the implied warranty of MERCHANTABILITY or
%% FITNESS FOR A PARTICULAR PURPOSE! 
%% User assumes all risk.
%% In no event shall IEEE or any contributor to this code be liable for
%% any damages or losses, including, but not limited to, incidental,
%% consequential, or any other damages, resulting from the use or misuse
%% of any information contained here.
%%
%% All comments are the opinions of their respective authors and are not
%% necessarily endorsed by the IEEE.
%%
%% This work is distributed under the LaTeX Project Public License (LPPL)
%% ( http://www.latex-project.org/ ) version 1.3, and may be freely used,
%% distributed and modified. A copy of the LPPL, version 1.3, is included
%% in the base LaTeX documentation of all distributions of LaTeX released
%% 2003/12/01 or later.
%% Retain all contribution notices and credits.
%% ** Modified files should be clearly indicated as such, including  **
%% ** renaming them and changing author support contact information. **
%%
%% File list of work: IEEEtran.cls, IEEEtran_HOWTO.pdf, bare_adv.tex,
%%                    bare_conf.tex, bare_jrnl.tex, bare_jrnl_compsoc.tex
%%*************************************************************************

% *** Authors should verify (and, if needed, correct) their LaTeX system  ***
% *** with the testflow diagnostic prior to trusting their LaTeX platform ***
% *** with production work. IEEE's font choices can trigger bugs that do  ***
% *** not appear when using other class files.                            ***
% The testflow support page is at:
% http://www.michaelshell.org/tex/testflow/



% Note that the a4paper option is mainly intended so that authors in
% countries using A4 can easily print to A4 and see how their papers will
% look in print - the typesetting of the document will not typically be
% affected with changes in paper size (but the bottom and side margins will).
% Use the testflow package mentioned above to verify correct handling of
% both paper sizes by the user's LaTeX system.
%
% Also note that the "draftcls" or "draftclsnofoot", not "draft", option
% should be used if it is desired that the figures are to be displayed in
% draft mode.
%
\documentclass[conference]{IEEEtran}
\usepackage{blindtext, graphicx}
% Add the compsoc option for Computer Society conferences.
%
% If IEEEtran.cls has not been installed into the LaTeX system files,
% manually specify the path to it like:
% \documentclass[conference]{../sty/IEEEtran}





% Some very useful LaTeX packages include:
% (uncomment the ones you want to load)


% *** MISC UTILITY PACKAGES ***
%
%\usepackage{ifpdf}
% Heiko Oberdiek's ifpdf.sty is very useful if you need conditional
% compilation based on whether the output is pdf or dvi.
% usage:
% \ifpdf
%   % pdf code
% \else
%   % dvi code
% \fi
% The latest version of ifpdf.sty can be obtained from:
% http://www.ctan.org/tex-archive/macros/latex/contrib/oberdiek/
% Also, note that IEEEtran.cls V1.7 and later provides a builtin
% \ifCLASSINFOpdf conditional that works the same way.
% When switching from latex to pdflatex and vice-versa, the compiler may
% have to be run twice to clear warning/error messages.






% *** CITATION PACKAGES ***
%
%\usepackage{cite}
% cite.sty was written by Donald Arseneau
% V1.6 and later of IEEEtran pre-defines the format of the cite.sty package
% \cite{} output to follow that of IEEE. Loading the cite package will
% result in citation numbers being automatically sorted and properly
% "compressed/ranged". e.g., [1], [9], [2], [7], [5], [6] without using
% cite.sty will become [1], [2], [5]--[7], [9] using cite.sty. cite.sty's
% \cite will automatically add leading space, if needed. Use cite.sty's
% noadjust option (cite.sty V3.8 and later) if you want to turn this off.
% cite.sty is already installed on most LaTeX systems. Be sure and use
% version 4.0 (2003-05-27) and later if using hyperref.sty. cite.sty does
% not currently provide for hyperlinked citations.
% The latest version can be obtained at:
% http://www.ctan.org/tex-archive/macros/latex/contrib/cite/
% The documentation is contained in the cite.sty file itself.






% *** GRAPHICS RELATED PACKAGES ***
%
\ifCLASSINFOpdf
  % \usepackage[pdftex]{graphicx}
  % declare the path(s) where your graphic files are
  % \graphicspath{{../pdf/}{../jpeg/}}
  % and their extensions so you won't have to specify these with
  % every instance of \includegraphics
  % \DeclareGraphicsExtensions{.pdf,.jpeg,.png}
\else
  % or other class option (dvipsone, dvipdf, if not using dvips). graphicx
  % will default to the driver specified in the system graphics.cfg if no
  % driver is specified.
  % \usepackage[dvips]{graphicx}
  % declare the path(s) where your graphic files are
  % \graphicspath{{../eps/}}
  % and their extensions so you won't have to specify these with
  % every instance of \includegraphics
  % \DeclareGraphicsExtensions{.eps}
\fi
% graphicx was written by David Carlisle and Sebastian Rahtz. It is
% required if you want graphics, photos, etc. graphicx.sty is already
% installed on most LaTeX systems. The latest version and documentation can
% be obtained at: 
% http://www.ctan.org/tex-archive/macros/latex/required/graphics/
% Another good source of documentation is "Using Imported Graphics in
% LaTeX2e" by Keith Reckdahl which can be found as epslatex.ps or
% epslatex.pdf at: http://www.ctan.org/tex-archive/info/
%
% latex, and pdflatex in dvi mode, support graphics in encapsulated
% postscript (.eps) format. pdflatex in pdf mode supports graphics
% in .pdf, .jpeg, .png and .mps (metapost) formats. Users should ensure
% that all non-photo figures use a vector format (.eps, .pdf, .mps) and
% not a bitmapped formats (.jpeg, .png). IEEE frowns on bitmapped formats
% which can result in "jaggedy"/blurry rendering of lines and letters as
% well as large increases in file sizes.
%
% You can find documentation about the pdfTeX application at:
% http://www.tug.org/applications/pdftex





% *** MATH PACKAGES ***
%
%\usepackage[cmex10]{amsmath}
% A popular package from the American Mathematical Society that provides
% many useful and powerful commands for dealing with mathematics. If using
% it, be sure to load this package with the cmex10 option to ensure that
% only type 1 fonts will utilized at all point sizes. Without this option,
% it is possible that some math symbols, particularly those within
% footnotes, will be rendered in bitmap form which will result in a
% document that can not be IEEE Xplore compliant!
%
% Also, note that the amsmath package sets \interdisplaylinepenalty to 10000
% thus preventing page breaks from occurring within multiline equations. Use:
%\interdisplaylinepenalty=2500
% after loading amsmath to restore such page breaks as IEEEtran.cls normally
% does. amsmath.sty is already installed on most LaTeX systems. The latest
% version and documentation can be obtained at:
% http://www.ctan.org/tex-archive/macros/latex/required/amslatex/math/





% *** SPECIALIZED LIST PACKAGES ***
%
\usepackage{algorithmic}
\usepackage{algorithm}
\floatname{algorithm}{Algoritmo}
\renewcommand{\algorithmicrequire}{\textbf{Input:}}
\renewcommand{\algorithmicensure}{\textbf{Output:}}
% algorithmic.sty was written by Peter Williams and Rogerio Brito.
% This package provides an algorithmic environment fo describing algorithms.
% You can use the algorithmic environment in-text or within a figure
% environment to provide for a floating algorithm. Do NOT use the algorithm
% floating environment provided by algorithm.sty (by the same authors) or
% algorithm2e.sty (by Christophe Fiorio) as IEEE does not use dedicated
% algorithm float types and packages that provide these will not provide
% correct IEEE style captions. The latest version and documentation of
% algorithmic.sty can be obtained at:
% http://www.ctan.org/tex-archive/macros/latex/contrib/algorithms/
% There is also a support site at:
% http://algorithms.berlios.de/index.html
% Also of interest may be the (relatively newer and more customizable)
% algorithmicx.sty package by Szasz Janos:
% http://www.ctan.org/tex-archive/macros/latex/contrib/algorithmicx/




% *** ALIGNMENT PACKAGES ***
%
%\usepackage{array}
% Frank Mittelbach's and David Carlisle's array.sty patches and improves
% the standard LaTeX2e array and tabular environments to provide better
% appearance and additional user controls. As the default LaTeX2e table
% generation code is lacking to the point of almost being broken with
% respect to the quality of the end results, all users are strongly
% advised to use an enhanced (at the very least that provided by array.sty)
% set of table tools. array.sty is already installed on most systems. The
% latest version and documentation can be obtained at:
% http://www.ctan.org/tex-archive/macros/latex/required/tools/


%\usepackage{mdwmath}
%\usepackage{mdwtab}
% Also highly recommended is Mark Wooding's extremely powerful MDW tools,
% especially mdwmath.sty and mdwtab.sty which are used to format equations
% and tables, respectively. The MDWtools set is already installed on most
% LaTeX systems. The lastest version and documentation is available at:
% http://www.ctan.org/tex-archive/macros/latex/contrib/mdwtools/


% IEEEtran contains the IEEEeqnarray family of commands that can be used to
% generate multiline equations as well as matrices, tables, etc., of high
% quality.


%\usepackage{eqparbox}
% Also of notable interest is Scott Pakin's eqparbox package for creating
% (automatically sized) equal width boxes - aka "natural width parboxes".
% Available at:
% http://www.ctan.org/tex-archive/macros/latex/contrib/eqparbox/





% *** SUBFIGURE PACKAGES ***
%\usepackage[tight,footnotesize]{subfigure}
% subfigure.sty was written by Steven Douglas Cochran. This package makes it
% easy to put subfigures in your figures. e.g., "Figure 1a and 1b". For IEEE
% work, it is a good idea to load it with the tight package option to reduce
% the amount of white space around the subfigures. subfigure.sty is already
% installed on most LaTeX systems. The latest version and documentation can
% be obtained at:
% http://www.ctan.org/tex-archive/obsolete/macros/latex/contrib/subfigure/
% subfigure.sty has been superceeded by subfig.sty.



%\usepackage[caption=false]{caption}
%\usepackage[font=footnotesize]{subfig}
% subfig.sty, also written by Steven Douglas Cochran, is the modern
% replacement for subfigure.sty. However, subfig.sty requires and
% automatically loads Axel Sommerfeldt's caption.sty which will override
% IEEEtran.cls handling of captions and this will result in nonIEEE style
% figure/table captions. To prevent this problem, be sure and preload
% caption.sty with its "caption=false" package option. This is will preserve
% IEEEtran.cls handing of captions. Version 1.3 (2005/06/28) and later 
% (recommended due to many improvements over 1.2) of subfig.sty supports
% the caption=false option directly:
%\usepackage[caption=false,font=footnotesize]{subfig}
%
% The latest version and documentation can be obtained at:
% http://www.ctan.org/tex-archive/macros/latex/contrib/subfig/
% The latest version and documentation of caption.sty can be obtained at:
% http://www.ctan.org/tex-archive/macros/latex/contrib/caption/




% *** FLOAT PACKAGES ***
%
%\usepackage{fixltx2e}
% fixltx2e, the successor to the earlier fix2col.sty, was written by
% Frank Mittelbach and David Carlisle. This package corrects a few problems
% in the LaTeX2e kernel, the most notable of which is that in current
% LaTeX2e releases, the ordering of single and double column floats is not
% guaranteed to be preserved. Thus, an unpatched LaTeX2e can allow a
% single column figure to be placed prior to an earlier double column
% figure. The latest version and documentation can be found at:
% http://www.ctan.org/tex-archive/macros/latex/base/



%\usepackage{stfloats}
% stfloats.sty was written by Sigitas Tolusis. This package gives LaTeX2e
% the ability to do double column floats at the bottom of the page as well
% as the top. (e.g., "\begin{figure*}[!b]" is not normally possible in
% LaTeX2e). It also provides a command:
%\fnbelowfloat
% to enable the placement of footnotes below bottom floats (the standard
% LaTeX2e kernel puts them above bottom floats). This is an invasive package
% which rewrites many portions of the LaTeX2e float routines. It may not work
% with other packages that modify the LaTeX2e float routines. The latest
% version and documentation can be obtained at:
% http://www.ctan.org/tex-archive/macros/latex/contrib/sttools/
% Documentation is contained in the stfloats.sty comments as well as in the
% presfull.pdf file. Do not use the stfloats baselinefloat ability as IEEE
% does not allow \baselineskip to stretch. Authors submitting work to the
% IEEE should note that IEEE rarely uses double column equations and
% that authors should try to avoid such use. Do not be tempted to use the
% cuted.sty or midfloat.sty packages (also by Sigitas Tolusis) as IEEE does
% not format its papers in such ways.





% *** PDF, URL AND HYPERLINK PACKAGES ***
%
%\usepackage{url}
% url.sty was written by Donald Arseneau. It provides better support for
% handling and breaking URLs. url.sty is already installed on most LaTeX
% systems. The latest version can be obtained at:
% http://www.ctan.org/tex-archive/macros/latex/contrib/misc/
% Read the url.sty source comments for usage information. Basically,
% \url{my_url_here}.





% *** Do not adjust lengths that control margins, column widths, etc. ***
% *** Do not use packages that alter fonts (such as pslatex).         ***
% There should be no need to do such things with IEEEtran.cls V1.6 and later.
% (Unless specifically asked to do so by the journal or conference you plan
% to submit to, of course. )

\usepackage[brazilian]{babel}
\usepackage[utf8]{inputenc}
\usepackage[T1]{fontenc}
\usepackage{fancyvrb}

% correct bad hyphenation here
\hyphenation{op-tical net-works semi-conduc-tor}


\begin{document}
%
% paper title
% can use linebreaks \\ within to get better formatting as desired
\title{WED-SQL (Rascunho)}


% author names and affiliations
% use a multiple column layout for up to three different
% affiliations
\author{\IEEEauthorblockN{Bruno Padilha, João E. Ferreira}
\IEEEauthorblockA{Departamento de Ciência da Computação\\
IME-USP\\
Rua do Matão 1010, 05508-090\\
São Paulo, SP, Brasil \\
brunopadilha@usp.br, jef@ime.usp.br}
\and
\IEEEauthorblockN{Calton Pu}
\IEEEauthorblockA{CERCS, Georgia Institute of Technology\\
266 Ferst Drive, 30332-0765\\
Atlanta, GA, USA\\
calton@cc.gatech.edu}}

% conference papers do not typically use \thanks and this command
% is locked out in conference mode. If really needed, such as for
% the acknowledgment of grants, issue a \IEEEoverridecommandlockouts
% after \documentclass

% for over three affiliations, or if they all won't fit within the width
% of the page, use this alternative format:
% 
%\author{\IEEEauthorblockN{Michael Shell\IEEEauthorrefmark{1},
%Homer Simpson\IEEEauthorrefmark{2},
%James Kirk\IEEEauthorrefmark{3}, 
%Montgomery Scott\IEEEauthorrefmark{3} and
%Eldon Tyrell\IEEEauthorrefmark{4}}
%\IEEEauthorblockA{\IEEEauthorrefmark{1}School of Electrical and Computer Engineering\\
%Georgia Institute of Technology,
%Atlanta, Georgia 30332--0250\\ Email: see http://www.michaelshell.org/contact.html}
%\IEEEauthorblockA{\IEEEauthorrefmark{2}Twentieth Century Fox, Springfield, USA\\
%Email: homer@thesimpsons.com}
%\IEEEauthorblockA{\IEEEauthorrefmark{3}Starfleet Academy, San Francisco, California 96678-2391\\
%Telephone: (800) 555--1212, Fax: (888) 555--1212}
%\IEEEauthorblockA{\IEEEauthorrefmark{4}Tyrell Inc., 123 Replicant Street, Los Angeles, California 90210--4321}}




% use for special paper notices
%\IEEEspecialpapernotice{(Invited Paper)}




% make the title area
\maketitle


\begin{abstract}
%\boldmath
%\blindtext[1]
Desenvolver e implementar um modelo de processo de negócios utilizando os conceitos do WED-flow resulta em um sistema de
informação dinâmico, simplificando o desenvolvimento incremental e adaptativo, além de reduzir significativamente a
complexidade no tratamento de exceções quando comparado aos arcabouços tradicionais (BPEL, álgebra de processos, Redes
de Petri e etc). No entanto, uma aplicação baseada no WED-flow precisa implementar uma camada de software em um nível
abaixo para fazer o controle transacional baseado em estados de dados (WED-states), assim como fornecer suporte ao tratamento
de exceções ou utilizar uma ferramenta como a WED-tool gerenciar esses mecanismos. A proposta da WED-SQL é fornecer um
arcabouço WED-flow distribuído e de alto desempenho para simplificar tanto o desenvolvimento quanto a implementação de tais
aplicações. Uma caracteristica peculiar da WED-SQL é que seu mecanismo de controle está integrado a um SGBD relacional e, com isso,
utiliza a linguagem SQL para a especificação das definições do WED-flow (WED-triggers, WED-transitions, WED-conditions, etc). 

\end{abstract}
% IEEEtran.cls defaults to using nonbold math in the Abstract.
% This preserves the distinction between vectors and scalars. However,
% if the journal you are submitting to favors bold math in the abstract,
% then you can use LaTeX's standard command \boldmath at the very start
% of the abstract to achieve this. Many IEEE journals frown on math
% in the abstract anyway.

% Note that keywords are not normally used for peerreview papers.
\begin{IEEEkeywords}
WED-flow, PostgreSQL, transações longas, ...
\end{IEEEkeywords}






% For peer review papers, you can put extra information on the cover
% page as needed:
% \ifCLASSOPTIONpeerreview
% \begin{center} \bfseries EDICS Category: 3-BBND \end{center}
% \fi
%
% For peerreview papers, this IEEEtran command inserts a page break and
% creates the second title. It will be ignored for other modes.
\IEEEpeerreviewmaketitle

\section{Introducão}

  Sistemas computacionais contemporaneos para o gerenciamento de processos de negócio estão sujeitos a constantes 
modificaçoes estruturais, ocasionadas tanto por execessões não previstas na fase de modelagem quanto para atender a novos 
requisitos. Ao longo do tempo, essas modificaçoes tendem a deteriorar a qualidade do código de tais sistemas, aumentando 
exponencialmente seu custo de manutenção além de, eventualmente, comprometer seu desempenho.
\\
\indent  Os modelos clássicos para especificação de processos de negócios, dentre eles o WS-BPEL, algebra de processos e redes
de petri, procuram capturar interações, relações e o comportamento entre processos, muitas vezes negligenciando a importância
dos dados no projeto. Nenhuma dessas ferramentas, no entanto, é exatamente adequada para modelar processos que necessitam
ser modificados frequentemente, o que implica em aumento exponencial no custo e na complexidade do projeto.
\\
\indent  O modelo WED-flow\cite{FTPM10}, como uma alternativa aos modelos clássicos, captura não só a dinâmica da interação entre processos
como também, com igual importância, os dados gerados assim como os eventos que implicam em alterações desses dados.
Com isso, o WED-flow é capaz de construir um workflow de estados de dados com suas respectivas condições de transição 
entre os mesmos. E assim, agregando propriedades de recuperação transacional a esse workflow, prover mecanismos para o 
tratamento de exceções em tempo real.
\\
\indent  Ao utilizar a abordagem WED-flow para modelar um processo de negócio, modificações nos processos são traduzidas em novos
estados de dados e novas regras, ou novos processos, se traduzem em condições para se atingir um determinado estado de 
dados. Essa versatilidade permite capturar a natureza de processos verdadeiramente dinâmicos, simplificando a evolução 
incremental tanto do modelo de negócio quanto do software que o implementa.
\\
%problemas da implementacao e justificativa do WED-SQL
%tolerancia a falhas
\indent Para que um modelo de processo de negócio baseado no WED-flow seja implementado em software, é preciso também
implementar as definições do WED-flow\cite{ICWS12}, que são a base para o tratamento de exceções e para o controle transacional.  
Ao invés de incorporar essas definições a cada software que implemente um modelo WED-flow, uma maneira mais eficiente de o fazer
é construir um arcabouço para padronizar a implementação do WED-flow em software. Nesse contexto, esse trabalho apresenta
o arcabouço WED-SQL.
\\
\indent O WED-SQL foi construido dentro de um SGBD relacional e com isso utiliza a linguagem SQL para definir propriedades
e controles de fluxo de um modelo WED-flow, além de contar com um robusto aparato transacional e ser altamente tolerante
a falhas. Por utilizar como base tecnologias consolidadas e amplamente difundidas no universo da computação, a WED-SQL visa 
disseminar a adoçao do paradigma WED-flow na modelagem de processos de negócio , fornecendo uma ferramenta confiável, facilmente
escalável, simples de ser utilizada e que permita a flexibilidade exigida na especificação de controle de fluxo dos processos
de negócio modernos.
\\
\indent O restante desse artigo segue a seguinte estrutura: ...
  

\section{WED-SQL: Visão geral e arquitetura}
%\blindtext
/* Incluir um resumo das definições do WED-flow ? */\\
Com o objetivo de  ser utilizada em um ambiente distribuido, por exemplo por meio de web-services, e também para contemplar 
o controle de transações longas \cite{MOLINA}, o WED-SQL foi construido utilizando a arquitetura cliente-servidor. A componente
servidor, ou WED-server, é responsável por fazer o controle transacional de acordo com as WED-conditions definidas
para um determinado WED-flow, disparando as WED-triggers conforme necessário. Já a componente cliente, ou WED-worker (
quem sabe até WED-service ?), é quem efetivamente realiza as WED-transitions.
\\
\subsection{WED-worker e WED-server}
O WED-server é, basicamente, uma extensão para o sgbd Postgresql composta de triggers, tabelas de controle e "stored
procedures". A escolha do Postgresql como base do WED-server se deu por diversos motivos, tais quais, ser de código aberto, 
implemetação modular baseada em catálogs (system catalog driven), facilidade de extender suas funcionalidades por meio de
código C e tambem em liguagens de alto nível como Python e Perl. Por ser de código aberto, a licensa do Postgresql garante
que o software possa ser modificado e redistribuido, além de que, o acesso ao código fonte possibilita uma melhor compreensão
dos mecanismos interno, garantido uma implementação mais robusta e eficiente do paradigma WED-flow. O postgresql armazena seus
dados de controle em tabelas que são acessíveis aos usuários, que são chamadas de catálogos de sistema, viabilizando a 
criação de modulos para extender suas funcionalidades. Dados de transações podem ser facilmente obtidos por meio desses 
catálogos de sistema, o que é partircularmente útil no caso do WED-server. Dada a natureza dinâmica do modelo WED-flow,
a utilização de uma linguagem de programação de alto nível, como Python, é fundamental para que a expressividade desse
modelo seja implementada de forma plena. Do ponto de vista de implementação, o WED-server contém código Python, SQL, C e 
aproveita o arcabouço transacional clássico (propriedades ACID, controle de concorrência e etc) fornecido pelo Postgresql.
\\
\indent Já o WED-worker tem a função de conectar-se ao WED-server para efetuar as WED-transitions. Cada WED-transition é 
associada a um ou mais WED-workers, dependendo da demanda de trabalho gerada pelo WED-server, e cada WED-worker é especializado
em realizar uma WED-transition específica. A quantidade máxima de WED-workers trabalhando simultaneamente é limitada
apenas pela quantidade de conecções que um dado WED-server é capaz de manter abertas.

\subsection{Estrutura de dados}

\begin{figure}[!t]
\centering
\includegraphics[width=2.5in]{ER.png}
\caption{Diagrama ER do WED-server}
\label{fig_er}

\end{figure}
 O diagrama Entidade-Relacionamento na figura~\ref{fig_er} representa o modelo de dados gerenciado internamente pelo 
WED-server. Vale notar que, devido a flexibilidade de modificaçao da estrutura de dados exigida para representar os WED-states, 
não é possível capturar todos os aspectos do paradigma WED-flow por meio de um diagrama Entidade-Relacionamento clássico. Por exemplo,
o relacionamento entre as entidades WED\_attr e WED\_flow é gerenciado pelo WED-server, ou seja, não há chaves indicando
a relação entre essas duas entidades. A principal razão disso é pertimir adicionar ou remover WED-attributes em tempo de execução, 
o que demanda modificações na estrutura de dados que, por sua vez, podem ser efetuadas de maneira ininterrupta.
\\
\indent Do relacionamento entre as entidades WED\_flow e WED\_trig resultam a entidade fraca Job\_Pool, que contém as WED-transitions
a serem executadas, e a entidade fraca WED\_trace, que contém os registros de execução de cada instância dos WED-flows.

\subsection{Tabelas de controle}

O WED-server utiliza cinco tabelas para definir regras, propriedades e controlar o fluxo de execução dos WED-flows: WED\_attr,
WED\_flow, WED\_trace, WED\_trig e Job\_pool.
\\
\indent  Os WED-attributes são definidos na tabela WED\_attr por um nome e, opcionalmente, por um valor padrão representados 
pelas colunas "aname" e "adv" respectivamente. Cada WED-attribute é identificado univocamente por seu nome, que também é 
a chave primária da tabela. Ao criar-se um novo WED-attribute, ou seja, ao inserir-se uma nova linha na tabela WED\_attr, 
uma coluna de mesmo nome será automaticamente criada na tabela WED\_flow.
\\
\indent A tabela WED\_trig contém as WED-triggers, ou seja, cada entrada representa a associação de uma WED-condition com uma 
WED-transition. Possui os seguintes atributos:

\begin{itemize}
\item $tgid$, é o identificador único de uma WED-trigger;
\item $tgname$, atributo opcional utilizado para dar um nome à WED-trigger;
\item $enabled$, permite que a WED-trigger seja desativada;
\item $trname$, identificador único da WED-transition associada.
\item $cname$, atributo opcional que pode ser utilizado para dar um nome à WED-condition associada;
\item $cpred$, predicado da WED-condition associada. Aceita qualquer predicado valido na claúsula WHERE em SQL;
\item $cfinal$, utilizada para indicar qual é a condição final. Embora apenas uma WED-condition possa ser marcada como final,
           o operado lógico \emph{OU} pode ser utilizado em seu predicado para definir-se multiplos WED-states finais.
\item $timeout$, tempo limite para que um WED-worker finalize a WED-transition representada por "trname".

\end{itemize}

O histórico de execução das WED-transition fica armazenado na tabela WED\_trace, que possui os seguintes atributos:
\begin{itemize}  
\item $wid$, referência ao identificador da instância do WED-flow;
\item $state$, WED-state, em formato json, que disparou as WED-transitions listadas em "trf";
\item $trf$, lista de WED-transitions disparadas pelo WED-state em "state";
\item $trw$, WED-transition que gravou o WED-state representado por "state". Um valor nulo indica que esse registro representa o WED-state inicial;
\item $status$, indica o tipo do WED-state representado em "state". Os possíveis valores são "F","E" ou "R" que indicam que o WED-state
           é final, excessão ou regular, respectivamente.
\item $tstmp$, indica o momento em que ocorreu o registro. Pode se recuperar a história de execução de uma instância ordenando-se
          as entradas nessa tabela por essa coluna.
\end{itemize}
  A tabela Job\_Pool contém entradas referentes as WED-transitions pendentes que precisam ser executadas pelos WED-workers.
Suas colunas são:
\begin{itemize}  
\item $wid$, referência ao identificador da instancia do WED-flow;
\item $tgid$, referência a WED-trigger que disparou a WED-transition "trname";
\item $trname$, nome da WED-transition a ser executada;
\item $lckid$, parâmetro opcional que pode ser utilizado pelos WED-workers para se identificarem. Futuramente, poderá ser utilizado
          para fins de autenticação e validação dos WED-workers;
\item $timeout$, tempo limite para a execução da WED-transition (tempo limite da transação). É uma cópia da coluna de mesmo
            nome da tabela WED\_trig;
\item $payload$, WED-state, em formato json, que disparou a referida WED-transition. Pode ser utilizado, por exemplo, por
            WED-workers que executam WED-transitions associadas a WED-conditions que possuem o operador lógico "OR" em 
            seu predicado.
\end{itemize}
  Finalmente, a tabela WED\_flow é o ponto de entrada para inicializar-se uma nova instância. Essa tabela é criada dinâmicamente
de acordo com os WED-attributes definidos na tabela WED\_attr. Cada entrada em WED\_attr corresponde a uma coluna em WED\_flow.
Suas entradas são o WED-state atual de cada instância, ou seja, são tuplas formadas por um identificador, representado
na coluna "wid", e os WED-attributes. 

\section{Funcionamento}
\begin{figure}[!t]
\begin{Verbatim}[fontsize=\tiny]
BEGIN;

INSERT INTO wed_attr (aname, adv) VALUES ('a1','ready');
INSERT INTO wed_attr (aname) VALUES ('a2'),('a3');

INSERT INTO wed_trig (tgname,trname,cname,cpred,timeout) 
  VALUES ('t1','tr_a2','c1', $$a1='ready' and (a2 is null)$$, '3d18h');
INSERT INTO wed_trig (tgname,trname,cname,cpred,timeout) 
  VALUES ('t2','tr_a3','c2', $$a1='ready' and (a3 is null)$$, '00:00:30');
INSERT INTO wed_trig (tgname,trname,cname,cpred,timeout) 
  VALUES ('tf','tr_final','cf', $$a1='ready' 
          and (a2 is not null) 
          and (a3 is not null)$$, '00:00:10');
INSERT INTO wed_trig (cpred,cfinal) VALUES ($$a1 <> 'ready'$$, True);

COMMIT;
\end{Verbatim}
\caption{Exemplo de definição de um novo WED-flow}
\label{fig_wf}
\end{figure}
 Embora seja possível definir multiplos WED-flows em uma única base de dados, o WED-server permite que cada um deles 
esteja localizado em uma base de dados distinta, de acordo com um determinado significado semântico. Com isso também é 
possível isolar atributos que não devem ser compartilhados entre diferentes WED-flows. 
\\
\indent Para se criar um novo WED-flow é preciso definir um conjunto de WED-atributes e um conjunto de WED-triggers associando
WED-transitions à WED-conditions. Essa definição, por ora, é expressa em SQL (futuramente em WSQL). Veja um exemplo na
figura~\ref{fig_wf}.
\\
\indent Note que os valores dos predicados para as WED-conditions, representados por meio da coluna "cpred", tem a mesma sintaxe
utilizada para expressar as restrições de uma cláusula WHERE em SQL. De fato, a expressão em "cpred" será utilizada as-is
pelo WED-server para disparar as WED-transitions. Ao utilizar-se duplo \$ para delimitar essa expressão, elimina-se a necessidade
de "escapar" as aspas simples ou outros caracteres especiais.
\\
\indent  Note também que a condição final de um WED-flow é declarada nessa mesma tabela WED\_trig, embora de modo simplificado.
São necessários apenas o predicado em "cpred" e o valor Verdade em "cfinal". Caso não haja uma condição final na tabela
WED\_trig, todas as instâncias desse WED-flow terminarao em um WED-state de excessão.
\\
\indent É recomendado encapsular a definição de um WED-flow em uma única transação, uma vez que, no caso de um erro de sintaxe
na definição de uma WED-trigger, por exemplo, seria necessário remover manualmente do WED-server as definições executadas
até o momento do erro.

\subsection{Definindo os WED-workers}
Como mencionado anteriormente, quem executa as WED-transitions disparadas pelo WED-server são os WED-workers. Sendo assim,
é necessário criar esses WED-workers e associá-los às respectivas WED-transitions. É recomendado ter ao menos um WED-worker
associado a cada WED-transition.
\\
\indent Um WED-worker é uma aplicação cliente do WED-server e, por esse motivo, pode ser escrito em qualquer linguagem de progra-
mação que tenha suporte para conectar-se ao sgbd PostgreSQL e que implemente o protocolo de comunicação do WED-server (veja
o capitulo ???). No escopo deste trabalho os WED-workers são escritos na linguagem Python utilizando-se o pacote BaseWorker, 
que acompanha o WED-SQL. A figura~\ref{fig_ww} ilustra a definição de um novo WED-worker:

\begin{figure}[!t]
\begin{Verbatim}[fontsize=\tiny]
from BaseWorker import BaseClass
import sys

class MyWorker(BaseClass):
    
    #  trname and dbs variables are static in order to conform 
    #with the definition of wed_trans()
        
    trname = 'tr_aaa'
    dbs = 'user=aaa dbname=aaa application_name=ww-tr_aaa'
    wakeup_interval = 5
    
    def __init__(self):
        super().__init__(MyWorker.trname,MyWorker.dbs,MyWorker.wakeup_interval)
    
    # Compute the WED-transition and return a string as the new WED-state, 
    #using the SQL SET clause syntax. Return None to abort transaction
    def wed_trans(self,payload):
        print (payload)
        
        return "a2='done', a3='ready', a4=(a4::integer+1)::text"
        #return None
        
w = MyWorker()

try:
    w.run()
except KeyboardInterrupt:
    print()
    sys.exit(0)
\end{Verbatim}
\caption{Exemplo de definição de um novo WED-worker}
\label{fig_ww}
\end{figure}

O primeiro passo é importar a classe abstrata \emph{BaseClass} do pacote \emph{BaseWorker}. Essa classe implementa a lógica do protocolo
de comunicação com o WED-server e também gerencia as conecções com o mesmo. 
\\
\indent  O proximo passo é de fato criar o WED-worker, definindo uma nova classe concreta que implemente o método wed\_trans() e
defina os seguintes atributos utilizados para inicializar a \emph{BaseClass}:
\begin{itemize}
\item \emph{trname}: nome da WED-transition que será executada por esse WED-worker. Precisa ser o mesmo utilizado na definicão
           do WED-flow;
\item \emph{dbs}: parametros da conecção com o WED-server no formato aceito pelo driver \emph{psycopg2}. No mínimo devem ser especificados
        o usuário, o nome da base de dados onde o WED-flow foi carregado e o nome do WED-worker por meio do campo \emph{"application\_name"};
\item \emph{wakeup\_interval}: é o intervalo de tempo no qual o WED-worker fica suspenso esperando por uma notificação do WED-server (veja capitulo ??);
\end{itemize}
  O método wed\_trans() recebe como parametro o WED-state da instância do WED-flow que disparou a transação, e retorna uma 
string com os novos valores dos WED-attributes dessa mesma instância. A sintaxe utilizada para esse valor de retorno deve
ser a mesma utilizada em uma cláusula SET de uma cláusula UPDATE em SQL. O valor "None" pode ser retornado para abortar
a transação. Essa classe concreta que implementa o WED-worker deve então ser instanciada e executada, invocando-se o método run().

\section{Algoritmos}
Após definir-se e carregar-se um WED-flow em uma base de dados do WED-server e inicializar-se seus respectivos WED-workers, uma
nova instância é inicializada inserindo-se um WED-state na tabela WED\_flow. Essa nova instância receberá um identificador
único que será utilizado tanto para o registro de sua história de execução, na tabela WED\_trace, quanto para o controle
transacional das WED-transitions. Se os valores padrao definidos para os WED-attributes, por meio da coluna "adv", na 
tabela WED\_attr forem os valores de um WED-state inicial, basta executar o comando abaixo para se inicializar uma nova
instância:
\begin{Verbatim}[fontsize=\small]

     INSERT INTO wed_flow DEFAULT VALUES;
\end{Verbatim}
  Para cada nova instância inserida na tabela wed\_flow, o WED-server irá comparar os predicados das WED-conditions definidos
na tabela wed\_trig com o WED-state representado na nova instância. Para cada condição satisfeita a respectiva WED-transition
será disparada na forma de uma entrada na tabela job\_pool e uma notificação será enviada ao WED-worker responsavel
por executá-la. Além disso, serão adicionadas entradas na tabela wed\_trace registrando os eventos ocorridos. O WED-server
ficará então aguardando até que uma nova instância ou uma WED-transition seja iniciada. Vale notar que WED-states iniciais 
que não sejam finais e que não disparem ao menos uma WED-transition serão rejeitados pelo WED-server.
\\
\indent  Na sequencia, os WED-workers podem atuar de dois modos distintos: imediatamente ao receber uma notificação do WED-server, ou quando o limite
de tempo de espera por notificações termina (atributo wakeup\_interval) e, nesse caso, é preciso consultar a tabela job\_pool em
busca de WED-transitions pendentes. Independentemente do modo de atuação e antes de inicializar a transação, cada WED-worker 
precisa solicitar ao WED-server uma trava (advisory\_lock) na WED-transition e em qual instância do WED-flow será executada a transação. 
Essa trava é necessária tanto para avisar a outros WED-workers que estejam trabalhando na mesma WED-transition que não a 
executem para a mesma instância, quanto para que o WED-server possa controlar o tempo limite da execução de cada WED-transition. 
Não é possivel executar uma WED-transition sem que o respectivo WED-worker consiga obter essa trava previamente.
\\
\indent De posse da referida trava, um WED-worker terá que finalizar a transação dentro do tempo limite de execução da WED-transition.
Essa transação é finalizada atualizando-se o WED-state atual da instância em que está sendo executada a WED-transition por meio
de um UPDATE na tabela wed\_flow.
\\
\indent Quando uma instância é atualizada, o WED-server novamente irá verificar se esse novo WED-state dispara novas WED-transitions
e, em caso afirmativo, registrá-las na tabela job\_pool, além que registrar os novos eventos em wed\_trace. Caso esse novo WED-state
satifaça a condição final e não haja nenhuma WED-transition pendente, essa instancia é marcada como finalizada e não poderá ser
modificada, a menos que sejam inseridos novos WED-attributes ou que sejam modificadas ou adicionadas novas WED-triggers. Em
caso de não houver WED-transitions pendentes e esse WED-state não for final, a instância será marcada como excessão e uma
entrada especial será adicionada a tabela job\_pool.
\begin{algorithm}
\caption{WED-server: disparo de WED-transitions}
\label{alg1}
\begin{algorithmic}[1]
\REQUIRE Instância $i$ de um WED-flow
\ENSURE \textbf{true} se a transação foi bem sucedida,\\
         \hspace{23pt}\textbf{false} em caso contrário
\STATE $L \leftarrow$ lista vazia de WED-transitions 
\STATE $s \leftarrow$ WED-state atual de $i$
\FOR{$tg$ in WED-triggers}
\STATE $c,tr \leftarrow$ $tg($WED-condition$)$,$tg($WED-transition$)$
\IF{$s$ satisfaz $c$}
\IF{$c$ é a condição final}
\STATE insira $\_FINAL$ em $L$
\ELSE
\STATE insira $tr$ em $L$
\ENDIF
\ENDIF
\ENDFOR
\IF{$L$ está vazia}
\IF{$i$ é uma nova instância}
\STATE Rejeite $i$ e aborte a transação
\RETURN \FALSE
\ELSIF{$i$ \textbf{NÃO} possui WED-transitions pendentes}
\STATE Maque $i$ como WED-state de exceção
\STATE Insira \_EXCPT em Job\_Pool
\ENDIF
\STATE Atualize $i$ em WED\_trace
\RETURN \TRUE
\ELSIF{$L$ contém $\_FINAL$}
\IF{$i$ possui WED-transitions pendentes}
\STATE Rejeite $i$ e aborte a transação
\RETURN \FALSE
\ELSE
\STATE Marque $i$ como WED-state final
\STATE Atualize $i$ em WED\_trace
\RETURN \TRUE
\ENDIF
\ELSE
\FOR{$tr$ em $L$}
\STATE Insira $tr$ na tabela Job\_Pool
\STATE Envie notificação para o WED-worker que executa $tr$
\RETURN \TRUE
\ENDFOR
\ENDIF

\end{algorithmic}
\end{algorithm}

\begin{algorithm}
\caption{WED-worker}
\label{alg1}
\begin{algorithmic}[1]
\REQUIRE Nome da WED-transition: \emph{trname}
\ENSURE \textbf{true} se a transação foi bem sucedida,\\
         \hspace{23pt}\textbf{false} em caso contrário
\LOOP
\STATE $n \leftarrow$ Valor nulo
\WHILE{$n$ é nula}
\STATE $n \leftarrow espera\_notificacao(trname,wkup)$
\IF [nenhuma notificação recebida após esperar $wkup$ segundos]{$n$ é nula}
\STATE $n \leftarrow $ busque por WED-transitions $trname$ pendentes no WED-server
\ENDIF
\ENDWHILE
\STATE Inicia a transação 
\IF {obter\_trava($n$)} 
\STATE Execute a WED-transition $trname$ para uma determinada instância $i$ do WED-flow
\STATE Finalize a transação
\RETURN \TRUE
\ELSE
\STATE Aborte a transação
\RETURN \FALSE
\ENDIF  
\ENDLOOP

\end{algorithmic}
\end{algorithm}

\section{Gerenciamento transacional}
Cada instância de um WED-flow pode ser vista como uma Transação Longa (LLT) e, de acordo com ~\cite{SGD87},
suas WED-transitions podem ser vistas como passos Saga. Sendo assim, é função do WED-server garantir a integridade dos estados
de dados, manter a consistência transacional e garantir que todo WED-flow termine ou em um estado final ou em um estado
de excessão. Nesse caso, deve permitir que um passo compensatório possa ser executado. 
\par
Nas sessoes seguintes serao apresentados em detalhes os mecanismos de controle transacional utilizados e o protocolo de
comunicação entre multiplos WED-workers e um WED-server.

%DETALHES DE FUNCIONAMENTO (seriação,locks(),keep connection alive(pessimistic locking),excessoes, tras simultaneas,continuous query,protocolo de comunicaocao, balanço: coneccoes x demanda,payload pode nao ser o estado atual mvcc)
\subsection{Notificações}
%send notify vs continuous query
Ao modelar um sistema cujas operações principais são realizadas com base em eventos de dados, é preciso também conceber-se
algum tipo de mecanismo capaz de capturar tais eventos. Quando esses dados estão armazenados em um SGBD, muitas vezes o
único mecanismo de detecção disponível é continuamente realizar uma consulta (query) a cada intervalo de tempo. Esse modelo,
conhecido por \emph{Continuous Query}(ref...), possui alguns inconvenientes embora, muitas vezes, é a única solução disponível.
\par O primeiro problema com o \emph{Continous Query} é que é preciso manter uma conecção continuamente aberta com
o SGBD durante toda a execução do sistema, mesmo que ela fique ociosa em boa parte do tempo. Alternatvamente pode-se abrir e fechar essa conecção 
inumeras vezes em um curto periodo de tempo, nesse caso o custo da consulta é significativamente menor do que o custo de
se realizar esse procedimento. Em ambos os casos há um desperdício de recursos computacionais, assim como de consumo de energia, uma
vez que esse consulta será executada em \emph{spin lock}.
\par Outro detalhe desse modelo é que, por se tratar de uma consulta que é realizada a cada intervalo de tempo, há sempre um
atraso na detecção de mudanças nos estados de dados, que por sua vez pode acabar comprometendo a capacidade do sistema
de responder a eventos em tempo real.
%sleep
\par O SGBD PostgreSQL oferece um mecanismo que pode ser utilizado como uma alternativa ao modelo anterior: Notificações Assíncronas.
Por meio do comando "NOTIFY <canal>,<payload>", o SGDB pode notificar de modo assíncrono um cliente que tenha se registrado, via comando "LISTEN <canal>", 
em um canal de notificação específico. Além disso, também é possivel enviar uma mensagem ao cliente, por meio de "<payload>". Essa mensagem pode
ser, por exemplo, uma cadeia de characteres no formato \emph{JSON} que represente o estado de dados de uma instância de um WED-flow.
\par A grande vantagem do modelo de Notificações Assíncronas sobre o modelo de Continuous Query é que o cliente recebe a notificação em tempo real
da ocorrência do evento de dados, contanto que o mesmo esteja escutando no canal de notificação nesse momento. Além disso, devido ao fato de o cliente
não precisar verificar a cada certo intervalo de tempo se há alguma nova mensagem, ele pode esperar pela notifição "dormindo", ou seja, sua execução
pode ser suspensa enquanto não há uma nova mensagem, liberando os recursos da máquina para realizar outras tarefas assim como economizando energia.
\par No WED-SQL são utilizados os dois modelos de detecção de eventos de dados. Durante uma execução normal, o WED-server notifica os WED-workers
sempre que uma nova WED-trigger é disparada, ou seja, sempre que surge uma nova tarefa na tabela Job\_Pool. Além disso, a mensagem enviada aos WED-workers
é exatamente a nova tarefa que foi gerada, eliminando a necessidade de se fazer uma busca na tabela Job\_Pool. Cada WED-transition possui um canal exclusivo
de notificações identificado por seu nome definido na tabela WED\_trig. Os WED-workers registram-se em seu respectivo canal e suspendem sua operação até
que sejam acordados com uma notificação. Caso uma notificação seja enviada e não haja ninguem escutando em um determinado canal, a mesma será descartada.
\par Afim de garantir que todas as WED-transitions sejam eventualmente executadas, e de preferência sejam iniciadas no menor intervalo de tempo possível, os
WED-workers não dependem exclusivamente das notificações recebidas para funcionar. De fato, ao inicializar uma nova conecção com o WED-server, o primeiro
passo é procurar por WED-transitions pendentes de execução na tabela Job\_Pool, e só após entrar em estado de suspensão aguardando por notificações. Os 
WED-workers podem ainda ser configurados para "acordar" após um certo intervalo de tempo sem receber novas notificações e verificar por tarefas pendentes,
uma vez que, em se tratando de um sistema potencialmente distribuido, é possível que mensagens possam se perder por falhas na rede de comunicaçao ou por
outros eventos nao detectáveis tanto pelo WED-server quanto pelos WED-workers. 
%Cabe ao projetista garantir que uma trnas nao dependa de atributos que nao foram definidos na cond (attr de interesse). Se a trnas foi disparada é pq a cond for satis. nivel de isolamento permite ...
\par O WED-state recebido pelos WED-workers, seja por meio de notificação ou consultando a tabela Job\_Pool, não representa o estado atual da instância mas
sim o estado de dados no momento em que a WED-transition foi disparada. Sendo assim, dado que a condição de disparo de uma WED-transition é conhecida,
qual a razão de se enviar o estado de dados aos WED-workers ? Eficiência. No caso de WED-conditions definidas por uma conjunção de predicados, ou seja, 
utilizando o operador lógico \emph{OR}, pode ser necessário verificar os valores dos WED-attributes que dispararam a WED-transtion associada. Enviar o estado
de dados aos WED-workers elimina a necessidade de consultas à tabela WED-trace. É muito importante lembrar que, ao definir um WED-worker, não se deve 
considerar atributos que não foram definidos na WED-condition, uma vez que, além de ser um erro de projeto do WED-flow, outras transações executando
em paralelo podem alterá-los a qualquer momento. Cabe ao projetista do WED-flow garantir que uma WED-transition não dependa de WED-atributes que não
foram definidos na WED-condition associada. 


\subsection{Controle transacional e de concorrência}
%wed-trans, why lock ?, types of locks , why advisory(pg_row_locks)
%locks are needed for: 1-keep track of wed-trans; 2-concurrency control between wed-workers (pg_locks, select for update nowait)

Antes que uma WED-transition possa ser executada, é necessário que a respectiva tarefa esteja registrada na tabela Job\_Pool e que
o WED-worker que irá executá-la garanta exclusividade de execução. Isso é efeito requisitando uma trava ao WED-server na tarefa
especificada. Essa trava é necessária por dois motivos: permitir que o WED-server tenha controle sobre o tempo de vida da transação;
gerenciar conflitos entre transações concorrentes.

\par O modelo WED-flow prevê que cada WED-transition tenha um tempo limite de execução. Para atender a esse requisito, o WED-server
precisa identificar a transição que está sendo executada e monitorar o seu tempo de vida. Vale lembrar que a execução de uma
WED-transition está encpsulada em uma unica transação. Um modo de forçar as transações a se identificarem é exigir que as mesmas
solicitem uma trava exclusiva ao WED-server. Outra possível solução seria identificar a transação semanticamente de acordo
com o novo WED-state gerado, e nesse caso teriamos dois problemas: o WED-server apenas seria capaz de identificar a transação
no momento da atualização da instância, fazendo com que as transações pudessem ficar ativas muito além de seu tempo limite para
só então serem finalizadas, potencialmente limitando o tempo máximo de execução para todas as transações; seria necessário manter 
um conjunto de possíveis estados de dados pré-definidos, limitando a capacidade do modelo em lidar com exceções. Além do mais,
o WED-SQL utiliza essas travas de execução para resolver problemas de concorrência entre WED-workers e, com isso,
é a solução adotada para a identificação das transações. 


\par Cada WED-transition pode ter multiplos WED-workers identicos associados. Isso pode ser necessário para atender a demanda 
gerada por um WED-flow com muitas instâncias. Para evitar que mais de um WED-worker execute a mesma WED-transition na mesma
instância, o mesmo deve registrar interesse em executá-la por meio da requisição ao WED-server de uma trava exclusiva na linha
correspondente da tabela Job\_Pool, que é identificada por seu WED-flow id (wid) e a WED-trigger id (tgid) que a disparou.

\par O SGDB PostgreSQL oferece um mecanismo de travas explicitas para aplicaçoes que necessitam exercer um controle mais
refinado sobre os dados acessados. Essas travas podem ser utilizadas tanto a nível de tabelas quanto a nivel de tuplas, ou seja,
linhas ou registros das tabelas. No contexto do WED-SQL, apenas as travas a nivel de tuplas são utilizadas explicitamente.
\par A trava a nivel de tupla mais comumente utilizada é chamada \emph{FOR UPDATE}. Utilizada em conjunto com o comando SQL
\emph{SELECT}, essa trava, de acesso exclusivo, faz com que outras transações concorrentes tenham que esperar ou abortar
ao tentarem acessar a tupla bloqueada. Outro tipo de trava que também pode ser utilizada a nível de tupla são as 
\emph{Advisory Locks}, ou travas semnaticas. Para utilizar esse tipo de trava é preciso definir seu significado semântico na aplicação, uma vez que
o PostgreSQL não obriga que sua regra de acesso seja cumprida, ou seja, os registros não são realmente bloqueados. 
\par Por exemplo, no WED-server as WED-transitions pendentes são identificadas pelo par (wid,tgid). Esse identificador é
utilizado pelos WED-workers para requisitar uma \emph{Advisory Lock}, que será concedida apenas à primeira transação que
a solicitar para o dado registro. As demais transações que tentarem acessar o mesmo registro não conseguiram obter a trava,
e estarão livres para tentar executar outras WED-transitions pendentes, embora o registro não esteja de fato bloqueado. Vale
lembrar que, obter essa trava para a instância específica que será atualizada, é mandatório para completar a transação com sucesso.
\par Por que são utilizadas \emph{Advisory Locks} ao invés de \emph{FOR UPDATE} no WED-SQL ? Por dois motivos: primeiramente
pelo modo como o PostgreSQL funciona internamente e, em segundo lugar, pela natureza da operação de  transição de estados
no contexto do WED-SQL. 
\par O principal motivo para não utilizar travas do tipo \emph{FOR UPDATE} é que as informações sobre
quais tuplas estão bloqueadas são armazenadas em disco, e não em memória RAM. Com isso, é necessário utilizar um plugin
externo ao SGBD para recuperar tais informações, necessárias ao WED-server para realizar o controle do tempo de vida
de cada transação. Por outro lado, as informações sobre \emph{Advisory Locks} podem ser obtidas facilmente, por meio de
uma simples consulta a um catálogo interno do PostgreSQL. 
\par Outro motivo para não se utilizar \emph{FOR UPDATE} no controle transacional, é que o registro que seria bloqueado
não é o registro que será atualizado. As travas são obtidas na tabela Job\_Pool, porém as atualizações são executadas na
tabela WED\_flow. Desse modo é mais simples, seguro e eficiente utilizar travas semanticas no WED-server.




\subsection{Isolamento e paralelismo}
O PostgreSQL utiliza o modelo MVCC para manter a integridade dos dados. Isso significa que cada transação enxerga uma "foto"
do banco de dados no momento em que a mesma inicia, garantindo seu isolamento uma vez que não há compartilhamento de dados
parciais entre transações ativas. Nesse modelo de controle de concorrencia, travas de leitura não conflitam com travas de escrita.
\par O padrão ISO/ANSI SQL define quatro níveis de isolamento transacional. Em ordem de restrição: \emph{
Read Uncommited, Read Commited, Repeatable Read e Serializable}
\par No nível menos restritivo,\emph{Read Uncommited}, uma transação enxerga modificações realizadas por outras transações
concorrentes que ainda não finalizaram. Esse fenômeno é conhecido como \emph{dirty read}. No PostgreSQL, transações que executam
nesse nível de isolamento são na verdade executadas no nível \emph{Read Commited}, que é o nível mínimo de isolamento possivel
na arquitetura MVCC. De qualquer modo, o padrão permite que um SGBD execute suas transações em um nível mais restritivo
de isolamento do que o requisitado.
\par Transações que executam no nível \emph{Read Commited} podem enxergar modificações realizadas por transações concorrentes
que finalizem durante sua execução. Por exemplo, a leitura de uma mesma tupla duas ou mais vezes, dentro de uma mesma transação,
pode retornar valores diferentes. Esse fenômeno é conhecido por \emph{nonrepeatable read}. Esse é o nível padrao do PostgreSQL.
\par Ao contrário do nível anterior, no nível \emph{Repetable Read}, uma transação ativa não enxerga tuplas modificadas
por outras transações que finalizem previamente. Em caso de conflito, apenas uma transação finaliza e as outras são abortadas.
Ainda assim, é possível ocorrer um fenômeno chamado de \emph{phantom read}, no qual um conjunto de tuplas que satisfaça uma
determinada condição difere em duas execuções da mesma consulta na mesma transação. No PostgreSQL esse fenômeno não ocorre.
\par O último e mais restritivo nível de isolamento é o \emph{Serializable}. De acordo com o padrão, um conjunto de transações
concorrentes executando nesse nível deve produzir o mesmo resultado da execução sequencial em alguma ordem. Sendo assim, nenhum
dos fenômenos mencionados anteriormente ocorre. No PostgreSQL, esse nível de isolamento funciona exatamente como o \emph{Repeatable Read}
a menos da introdução de um mecanismo para detectar anomalias que poderiam levar a um resultado não condizente com todas
as possives execuções de um conjunto de transações em série.
\par No WED-server, assim como no PostgreSQL, o nível padrão de isolamento transacional é o \emph{Read Commited}. Isso só
é possível graças ao seu mecanismo de detecção de estados de dados que, valendo-se do modelo MVCC, gera uma nova versão das
tabelas relevantes toda vez que uma transação modifica uma instância de um WED-flow. Sendo assim, no caso de duas transações 
concorrentes, por exemplo, não é preciso que uma delas seja abortada, o que ocorreria em níveis mais restritos de isolamento.
Caso uma transação aborte, o WED-server descarta as alterações realizadas. Transações que operam em instância distintas 
são executadas totalmente em paralelo. Além disso, os WED-workers não exergarao as novas WED-transitions 
disparadas até que transação que escreveu o novo estado de dados finalize com sucesso.
\par WED-workers que trabalham simultaneamente na mesma instância executam em paralelo até o momento final, onde necessitam
atualizar o estado de dados dessa instância. Nesse momento, a execução se dará de modo sequencial, uma vez que o PostreSQL
irá conceder uma trava implicita para a tupla que será modificada. Vale notar que a atualização da instância deve ser a 
ultima operação executada pelo WED-worker, limitando o tempo de espera à quantidade de transações pendentes multiplicada 
pelo tempo de execução do comando SQL UPDATE. Caso contrário, a execução ocorre em paralelo (veja a figura~\ref{fig_wf}).
   
\begin{figure}[!t]
\centering
\includegraphics[width=2.5in]{wed-flow2.png}
\caption{Os WED-workers 1 e 2 trabalham na mesma instância e precisam sincronizar a atualização do WED-state na tabela
WED-flow}
\label{fig_wf}
\end{figure}
%escalabilidade (horizontal, vertical, consistencia transacional, pg_shard) 

% needed in second column of first page if using \IEEEpubid
%\IEEEpubidadjcol

% An example of a floating figure using the graphicx package.
% Note that \label must occur AFTER (or within) \caption.
% For figures, \caption should occur after the \includegraphics.
% Note that IEEEtran v1.7 and later has special internal code that
% is designed to preserve the operation of \label within \caption
% even when the captionsoff option is in effect. However, because
% of issues like this, it may be the safest practice to put all your
% \label just after \caption rather than within \caption{}.
%
% Reminder: the "draftcls" or "draftclsnofoot", not "draft", class
% option should be used if it is desired that the figures are to be
% displayed while in draft mode.
%
%\begin{figure}[!t]
%\centering
%\includegraphics[width=2.5in]{myfigure}
% where an .eps filename suffix will be assumed under latex, 
% and a .pdf suffix will be assumed for pdflatex; or what has been declared
% via \DeclareGraphicsExtensions.
%\caption{Simulation Results}
%\label{fig_sim}
%\end{figure}

% Note that IEEE typically puts floats only at the top, even when this
% results in a large percentage of a column being occupied by floats.


% An example of a double column floating figure using two subfigures.
% (The subfig.sty package must be loaded for this to work.)
% The subfigure \label commands are set within each subfloat command, the
% \label for the overall figure must come after \caption.
% \hfil must be used as a separator to get equal spacing.
% The subfigure.sty package works much the same way, except \subfigure is
% used instead of \subfloat.
%
%\begin{figure*}[!t]
%\centerline{\subfloat[Case I]\includegraphics[width=2.5in]{subfigcase1}%
%\label{fig_first_case}}
%\hfil
%\subfloat[Case II]{\includegraphics[width=2.5in]{subfigcase2}%
%\label{fig_second_case}}}
%\caption{Simulation results}
%\label{fig_sim}
%\end{figure*}
%
% Note that often IEEE papers with subfigures do not employ subfigure
% captions (using the optional argument to \subfloat), but instead will
% reference/describe all of them (a), (b), etc., within the main caption.


% An example of a floating table. Note that, for IEEE style tables, the 
% \caption command should come BEFORE the table. Table text will default to
% \footnotesize as IEEE normally uses this smaller font for tables.
% The \label must come after \caption as always.
%
%\begin{table}[!t]
%% increase table row spacing, adjust to taste
%\renewcommand{\arraystretch}{1.3}
% if using array.sty, it might be a good idea to tweak the value of
% \extrarowheight as needed to properly center the text within the cells
%\caption{An Example of a Table}
%\label{table_example}
%\centering
%% Some packages, such as MDW tools, offer better commands for making tables
%% than the plain LaTeX2e tabular which is used here.
%\begin{tabular}{|c||c|}
%\hline
%One & Two\\
%\hline
%Three & Four\\
%\hline
%\end{tabular}
%\end{table}


% Note that IEEE does not put floats in the very first column - or typically
% anywhere on the first page for that matter. Also, in-text middle ("here")
% positioning is not used. Most IEEE journals use top floats exclusively.
% Note that, LaTeX2e, unlike IEEE journals, places footnotes above bottom
% floats. This can be corrected via the \fnbelowfloat command of the
% stfloats package.

\section{Trabalhos Futuros}
%pgBouncer = breaks notify
%wasting connections
%pubsub (rabbitMQ, real event bus)
%pg process size in memory (~9MB)
%file descriptor per process
(linguagem, gerenciador de conecções, ...)

Na próxima fase desse projeto serão incorporadas duas funcionaliades essenciais para o sucesso do projeto WED-SQL: uma
linguagem declarativa para expressar as operações nativas do WED-flow e um módulo específico para gerenciar as conecções
com o WED-server.
\par Quanto à linguagem, a ideia é criar uma extensao da SQL que encapsule as operações nativas do WED-flow e integrá-la
ao interpretador do PostgreSQL, o qual será então responsável por traduzir e executar código SQL nativo e outras operações que,
nesse momento, são realizadas por meio de um conjunto de \emph{shell scripts}. Por meio dessa abstração, será possível 
deixar a sintaxe declarativa do WED-server mais próxima do modelo teórico, simplificando sua utilização.
\par Em Linux, um processo pode criar um clone de si mesmo para executar uma porção específica de código de modo independente.
O PostgreSQL, assim como muitos outros serviços que atendem à requisições externas, utiliza essa técnica sempre que 
for estabelecida uma nova conecçao com uma aplicação cliente. Ao analisar a utilização de recursos do sistema operacional
, é possível observar que cada novo sub-processo gerado ocupa cerca de 9MB de memória RAM. Como consequência, há um limite
físico no número de conecções que podem ser estabelecidas simultaneamente. No caso do WED-SQL, cada WED-worker precisa 
manter uma conecção ativa com o WED-server para receber as notificações
assíncronas em tempo real, além de estabelecer uma segunda conecção para efetivamente executar a WED-transition. Sendo
assim, para cada WED-worker ativo são criados dois sub-processos do PostgreSQL, o que acaba por consumir em média 18MB
de RAM.
\par As notificações assíncronas do PostgreSQL são baseadas em um modelo de comunicação conhecido por \emph{Publish-Subscribe},
onde quem envia uma mensagem não esta ciente da existência de quem a recebe. Sendo assim, é possível utilizar uma especie
de WED-worker específico que funcionaria com um detector de eventos em tempo real para receber e distribuir as mensagens
para os demais WED-workers, o qual utilizaria uma única conecção com o WED-server, efetivamente dobrando a quantidade
de WED-transitions simultâneas. Há diversas soluções prontas para implementar esse mecanismo, umas delas é o software 
RabbitMQ.


\section{Conclusão}
Simplificar a utilização e a implementação de modelos de processos de negócio desenvolvidos com o paradigma WED-flow.
%\blindtext





% if have a single appendix:
%\appendix[Proof of the Zonklar Equations]
% or
%\appendix  % for no appendix heading
% do not use \section anymore after \appendix, only \section*
% is possibly needed

% use appendices with more than one appendix
% then use \section to start each appendix
% you must declare a \section before using any
% \subsection or using \label (\appendices by itself
% starts a section numbered zero.)
%


%\appendices
%\section{Apd }
%\blindtext

% use section* for acknowledgement
\section*{Acknowledgment}


The authors would like to thank...


% Can use something like this to put references on a page
% by themselves when using endfloat and the captionsoff option.
\ifCLASSOPTIONcaptionsoff
  \newpage
\fi



% trigger a \newpage just before the given reference
% number - used to balance the columns on the last page
% adjust value as needed - may need to be readjusted if
% the document is modified later
%\IEEEtriggeratref{8}
% The "triggered" command can be changed if desired:
%\IEEEtriggercmd{\enlargethispage{-5in}}

% references section

% can use a bibliography generated by BibTeX as a .bbl file
% BibTeX documentation can be easily obtained at:
% http://www.ctan.org/tex-archive/biblio/bibtex/contrib/doc/
% The IEEEtran BibTeX style support page is at:
% http://www.michaelshell.org/tex/ieeetran/bibtex/
%\bibliographystyle{IEEEtran}
% argument is your BibTeX string definitions and bibliography database(s)
%\bibliography{IEEEabrv,../bib/paper}
%
% <OR> manually copy in the resultant .bbl file
% set second argument of \begin to the number of references
% (used to reserve space for the reference number labels box)
\begin{thebibliography}{1}

\bibitem{FTPM10}
João E. Ferreira, Osvaldo K. Takai, Simon Malkowski e Calton Pu. 
\emph{Reducing exception handling complexity in business process modeling and implementation:
the WED-flow approach.}, Em Proceedings of the 2010 international conference on
On the move to meaningful internet systems - Volume Part I, OTM’10, páginas
150–167. Springer-Verlag, 2010.

\bibitem{ICWS12}
João Eduardo Ferreira, Kelly Rosa Braghetto, Osvaldo Kotaro Takai e
Calton Pu.\emph{ Transactional recovery support for robust exception handling in business process
services.} Em Proceedings of the 19th International Conference on Web Services (ICWS), páginas
303–310, 2012

\bibitem{SGD87}
Hector Garcia Molina, Kenneth Salem.
\emph{Sagas}, Em Proceeding of the 1987 ACM SIGMOD International Conference on Management of Data, páginas 249-259, 1987

%\bibitem{IEEEhowto:kopka}
%H.~Kopka and P.~W. Daly, \emph{A Guide to \LaTeX}, 3rd~ed.\hskip 1em plus
%  0.5em minus 0.4em\relax Harlow, England: Addison-Wesley, 1999.

\end{thebibliography}

% biography section
% 
% If you have an EPS/PDF photo (graphicx package needed) extra braces are
% needed around the contents of the optional argument to biography to prevent
% the LaTeX parser from getting confused when it sees the complicated
% \includegraphics command within an optional argument. (You could create
% your own custom macro containing the \includegraphics command to make things
% simpler here.)
%\begin{biography}[{\includegraphics[width=1in,height=1.25in,clip,keepaspectratio]{mshell}}]{Michael Shell}
% or if you just want to reserve a space for a photo:

\begin{IEEEbiography}[{\includegraphics[width=1in,height=1.25in,clip,keepaspectratio]{picture.png}}]{John Doe}
\blindtext
nhaga
\end{IEEEbiography}

% You can push biographies down or up by placing
% a \vfill before or after them. The appropriate
% use of \vfill depends on what kind of text is
% on the last page and whether or not the columns
% are being equalized.

%\vfill

% Can be used to pull up biographies so that the bottom of the last one
% is flush with the other column.
%\enlargethispage{-5in}




% that's all folks
\end{document}


